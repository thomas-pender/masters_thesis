\documentclass[../../main]{subfiles}

\newendnotes{c}
\input{ch3notes.txt}
\setcounter{cnote}{\value{bnote}}
\newcommand{\Cnote}[1]{\cnote{#1}\big)}

\begin{document}

This third and final preliminary chapter briefly touches on association schemes, a fundamental abstract object used as a unifying tool across the otherwise disparate fields of combinatorics. It is composed of two sections. The first will introduce the basic ideas via strongly regular graphs, the simplest nontrivial example of an association scheme. The second moves on to consider these objects in general. Only enough theory is developed in order to be applied in later chapters.

 \section{Strongly Regular Graphs}
 
 This section introduces the so-called strongly regular graphs. Much attention has been given to these objects, both in excogitating their properties as well as answering existence and classification problems. Many of the resuts of this section will be generalized in the next section.
 
 \dinkus
 
 \subfile{./srg/srg}
 
 \section{Association Schemes}
 
 Here we generailze some the concepts of the previous section, and we will mostly draw from Chapter 2 of \cite{bannaialgebraic}, the standard reference on the subject.
 
 \dinkus
 
 \subfile{./definitions/definitions}
 
 \singlespace
 
 \addcontentsline{toc}{section}{Notes}
 \section*{\centering Notes}
 \thecnotes
 
 \doublespacing
 
 \biblio
\end{document}
