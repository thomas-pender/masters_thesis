%\providecommand{\main}{../../main}
\documentclass[../../main]{subfiles}

\newendnotes{a}
\input{ch1notes.txt}
\newcommand{\Anote}[1]{\anote{#1}\big)}

\begin{document}

This first preliminary chapter introduces incidence in the context of two different but related objects. The first section will discuss the fundamental incidence structure underlying our constructions, namely, the balanced incomplete block designs. The second and final section moves on to consider in brief error-correcting codes. 

 \section{\centering Balanced Incomplete Block Designs}
 
 This section presents some basic definitions and results about block designs that will be used throughout this work. Particular emphasis will be placed on matrix representations of such objects.
 
 \dinkus
 
 \subfile{./bibd/bibd}
 
 \section{\centering Error-Correcting Codes}
 
 In this section, the definitions of linear error-correcting codes will be given. We then move on to consider the famous generalized Hamming and simplex codes. As these are the only family of codes that we require, this section will be brief. The interested reader is refered to the standard references of \cite{pless-book} and \cite{error-correcting-codes-v1} for a greater exposition of the subject.
 
 \dinkus
 
 \subfile{./codes/codes}
 
 \singlespace
 
 \addcontentsline{toc}{section}{Notes}
 \section*{\centering Notes}
 \theanotes
 
 \doublespacing
 
 \biblio
\end{document}
