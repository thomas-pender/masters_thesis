\documentclass[../../main]{subfiles}

\newendnotes{a}
\input{intronotes.txt}
% \newcommand{\Anote}[1]{\anote{#1}\big)}
\newcommand{\Anote}[1]{\anote{#1}}

\begin{document}

This preliminary chapter introduces the objects with which we will be working.
We intend to move briskly, so only references to proofs and more in-depth
discussions will be given. The following sections will define in turn: balanced
incomplete block designs, error-correcting codes, weighing matrices and their
generalizations, and finally association schemes.

\fancyhf{}

\fancyhead[RO,LE]{\thepage}
\fancyhead[CO]{\S\thesection. Balanced Incomplete Block Designs}
\fancyhead[CE]{Chapter \thechapter. Introduction}

\section{\centering Balanced Incomplete Block Designs}

This section presents some basic definitions and results about block designs
that will be used throughout this work. Particular emphasis will be placed on
matrix representations of such objects. 

\dinkus

\subfile{./bibd/bibd}

\fancyhf{}

\fancyhead[RO,LE]{\thepage}
\fancyhead[CO]{\S\thesection. Error-Correcting Codes}
\fancyhead[CE]{Chapter \thechapter. Introduction}

\section{\centering Error-Correcting Codes}

In this section, the definitions of linear error-correcting codes will be given.
We then move on to consider the famous generalized Hamming and simplex codes. As
these are the only family of codes that we require, this section will be brief.
The interested reader is referred to the standard references of \cite{pless-book}
and \cite{error-correcting-codes-v1} for a greater exposition of the subject. 

\dinkus

\subfile{./codes/codes}

\fancyhf{}

\fancyhead[RO,LE]{\thepage}
\fancyhead[CO]{\S\thesection. Weighing Matrices and Their Generalizations}
\fancyhead[CE]{Chapter \thechapter. Introduction}

\section{\centering Weighing Matrices and Their Generalizations}

This section focuses on a third combinatorial configuration, namely, weighing 
matrices together with their generalizations. We begin with a brief look at
weighing matrices themselves before moving onto to consider two generalizations.
The first generalization is allowing the entries of a weighing matrix to be
chosen from a finite group. This idea is then synthesized with the ideas of
$\S1$. Finally, we allow the entries of a weighing matrix to be taken from sets
of indeterminates and consider the utility of such an approach. 

\dinkus

\subfile{./weighing/weighing}

\fancyhf{}

\fancyhead[RO,LE]{\thepage}
\fancyhead[CO]{\S\thesection. Association Schemes}
\fancyhead[CE]{Chapter \thechapter. Introduction}

\section{Association Schemes}

This fourth and final preliminary section briefly touches on association schemes, a fundamental abstract object used as a unifying tool across the otherwise disparate fields of combinatorics. 

\dinkus

\subfile{./schemes/schemes}

\singlespace

\fancyhf{}

\fancyhead[RO,LE]{\thepage}
\fancyhead[CO]{Notes}
\fancyhead[CE]{Chapter \thechapter. Introduction}

\addcontentsline{toc}{section}{Notes}
\section*{\centering Notes}
\theanotes

\doublespacing

\biblio
\end{document}