\documentclass[../../../main]{subfiles}

\begin{document}
% subsection %%%%%%%%%%%%%%%%%%%%%%%%%%%%%%%%%%%%%%%%%%%%%%%%%%%%%%%%%%%%%%%%%%%%%%%%%%%%%%%%%%%%%%%%%%%%%%%%%%%%%
\subsection{Generalized weighing matrices}

In the previous section, we defined a weighing matrix as a square matrix over $\{-1,0,1\}$. We then extended this definition to include those matrices over $0$ together with the complex $p$-th roots of unity. More generally, we can have weighing matrices over any finite group. 

Before we can do this, however, we need to extend the conjugate transpose\index{conjugate transpose} to group matrices. To accomplish this, let $A$ be some matrix over a finite group $G$, and define $\bar A$ by $\bar{A}_{ij}=A_{ij}^{-1}$, that is, the matrix obtained by taking the group inverse of the nonzero entries of $A$. Finally, define $A^* = \bar{A}^t$. We then have the following.
 
 \begin{defin}\label{group weighing}\index{generalized weighing matrix}
  Let $G$ be some finite group not containing the symbol $0$, and let $W$ be a $(0,G)$-matrix of order $v$. If $WW^* = kI_n$ modulo the ideal $\Z G$, then we say that $W$ is a generalized weighing matrix of order $v$ and weight $k$. We write $\gw(v,k;G)$ to denote this property.\Bnote{group-ring}
 \end{defin}

 \begin{ex}
  A real $\w(v,k)$ is a $\gw(v,k;C_2)$, and a $\bw(v,k;p)$ is a $\gw(v,k;C_p)$, were $C_p$ denotes the cyclic group of prime order $p$.
 \end{ex}
 
 \begin{ex}
  In the case that one has $\bw(v,k;C_n)$, where $n$ is composite, one does not in general have a generalized weighing matrix. Consider the $\bw(6,6;4)$ given by
  \begin{equation}
  \arraycolsep=1.25pt\def\arraystretch{0.625}
\left(\begin{array}{cccccc}
i & + & + & + & + & + \\
+ & i & - & + & - & + \\
+ & - & i & - & + & + \\
+ & + & - & i & + & - \\
+ & - & + & + & i & - \\
+ & + & + & - & - & i
\end{array}\right).
  \end{equation}
  Clearly, this is a Hadamard matrix; however, $\pm i$ each appear only once in the conjugate inner product between distinct rows, hence it is not a generalized weighing matrix.
 \end{ex}
 
 \begin{ex}\label{bgw(15,7,3)1}
  The following is a $\gw(15,7;C_3)$, where the nonzero elements are the logarithms of a generator of $C_3$.
  \begin{equation}\label{gw-ex}
  \arraycolsep=2.0pt\def\arraystretch{0.5}
\left(\begin{array}{ccccccccccccccc}
0&3&3&3&3&3&3&3&0&0&0&0&0&0&0\\
0&3&2&0&1&0&0&0&0&0&2&0&2&2&3\\
0&0&3&2&0&1&0&0&3&0&0&2&0&2&2\\
0&0&0&3&2&0&1&0&2&3&0&0&2&0&2\\
0&0&0&0&3&2&0&1&2&2&3&0&0&2&0\\
0&1&0&0&0&3&2&0&0&2&2&3&0&0&2\\
0&0&1&0&0&0&3&2&2&0&2&2&3&0&0\\
0&2&0&1&0&0&0&3&0&2&0&2&2&3&0\\
3&3&1&0&2&0&0&0&3&2&0&1&0&0&0\\
3&0&3&1&0&2&0&0&0&3&2&0&1&0&0\\
3&0&0&3&1&0&2&0&0&0&3&2&0&1&0\\
3&0&0&0&3&1&0&2&0&0&0&3&2&0&1\\
3&2&0&0&0&3&1&0&1&0&0&0&3&2&0\\
3&0&2&0&0&0&3&1&0&1&0&0&0&3&2\\
3&1&0&2&0&0&0&3&2&0&1&0&0&0&3
\end{array}\right)
  \end{equation}
 \end{ex}
 
 \dinkus
 
 % subsection %%%%%%%%%%%%%%%%%%%%%%%%%%%%%%%%%%%%%%%%%%%%%%%%%%%%%%%%%%%%%%%%%%%%%%%%%%%%%%%%%%%%%%%%%%%%%%%%%%%%%
 \subsection{Generalized Bhaskar Rao Designs}
 
 Our goal in introducing weighing matrices over arbitrary finite groups is to synthesize the ideas of weighing matrices and balanced incomplete block designs. We combine these concepts thus.
 
 \begin{defin}\label{gbrd definition}\index{generalized Bhaskar Rao design}
  Let $G$ be some finite group, and let $A$ be a $v \times b$ $(0,G)$-matrix such that
  \begin{equation}\label{gbrd-eq}
  AA^* = rI_v + \frac{\lambda}{|G|}\left(\sum_{g \in G}g\right)(J_v - I_v),
  \end{equation}
  for some positive integers $r$ and $\lambda$, and such that there are $k$ non-zero entries in every column. We then say that $A$ is a {\it generalized Bhaskar Rao design} (henceforth $\gbrd$), and we write $\gbrd(v,k,\lambda;G)$ to denote this property. If we need to stress the remaining parameters, then we write $\gbrd(v,b,r,k,\lambda;G)$.
 \end{defin}
 
 Often it is helpful to give a combinatorial definition of $\gbrd$s that is equivalent to the one just given. Again let $A$ be a $v \times b$ $(0,G)$-matrix that has $k$ nonzero entries in every column. If the multisets $\{A_{i\ell}A_{j\ell}^{-1} : A_{i\ell} \neq 0 \neq A_{j\ell} \text{ and } 0 
 \leq \ell < b\}$, for $i,j \in \{0,\dots,v-1\}$, $i \neq j$, have $\lambda/|G|$ copies of every group element in $G$, then we say that $A$ is a $\gbrd(v,k,\lambda;G)$.
 
 A few things are rather immediate. If $\check A$ denotes the matrix obtained from $A$ by changing each non-zero entry to 1, then condition (\ref{gbrd-eq}) implies that $\check A$ is a $\bibd$. Conversely, a $\bibd$ is a $\gbrd$ over the trivial group $\{1\}$.
 
 Evidently, Fisher's inequality applies, hence $b \geq v$. We single out the extremal case of Fisher's inequality again.
 
 \begin{defin}\label{bgw definition}\index{balanced generalized weighing matrix}\index{generalized Hadamard matrix}
 A {\it balanced generalized weighing matrix} is a $\gbrd(v,b,r,k,\lambda;G)$ in which $v = b$ (equiv. $k = r$). We use the denotation $\bgw(v,k,\lambda;G)$. A $\bgw(v,k,\lambda;G)$ in which $v = k$ is called a {\it generalized Hadamard matrix}, and we denote this as $\gh(G,\lambda)$ where $\lambda = v/|G|$. If $G = \mathrm{EA}(q)$, the elementary abelian group\Bnote{ea-group}\index{elementary abelian group} of order $q$, then we write $\mathrm{GH}(q,\lambda)$ instead.
 \end{defin}
 
 \begin{ex}
  The generalized weighing matrix (\ref{gw-ex}) is a $\bgw(15,7,3;C_3)$.
 \end{ex}
 
 \begin{ex}
  Let $G=\langle \alpha,\beta : \alpha^2=\beta^2=1, \alpha\beta=\beta\alpha \rangle$. Then
  \begin{equation}\label{gh-ex}
  \arraycolsep=1.25pt\def\arraystretch{0.625}
   \left(\begin{array}{rrrr}
    1 & 1 & 1 & 1 \\
    1 & \alpha & \beta & \alpha\beta \\
    1 & \beta & \alpha\beta & \alpha \\
    1 & \alpha\beta & \alpha & \beta
   \end{array}\right)
  \end{equation}
  is a $\gh(4,1)$.
 \end{ex}
 
 \begin{ex}
  A $\bgw(v,k,\lambda;\{-1,1\})$ is a {\it balanced weighing matrix}\index{weighing matrix!balanced}. The weighing matrix (\ref{signed-fano}) is balanced.
 \end{ex}
 
 \dinkus
 
 % subsection %%%%%%%%%%%%%%%%%%%%%%%%%%%%%%%%%%%%%%%%%%%%%%%%%%%%%%%%%%%%%%%%%%%%%%%%%%%%%%%%%%%%%%%%%%%%%%%%%%%%%
 \subsection{Properties and Simple Constructions}
 
 Since A GBRD ultimately yields a BIBD, it follow that the necessary conditions of Proposition \ref{prop-bibd-params} also hold for the parameters of GBRDs. However, since we are now dealing with group matrices that are balanced with respect to the group, the next result is clear.
 
 \begin{prop}\label{gbrd epimorphism}
  Let $A$ be a $\gbrd(v,k,\lambda;G)$, and let $\phi: G \rightarrow H$ be some group epimorphism. Then $(\phi(A_{ij}))$ is a $\gbrd(v,k,\lambda)$ over $H$ with the same parameters.
 \end{prop}
 
 Our work from the previous section, namely, Lemma \ref{butson lemma} yields the following.
 
 \begin{prop}
  A Butson weighing matrix $W$ over the $p$-th roots of unity, where $p$ is a prime, is a $\bgw$ if and only if $(|W_{ij}|)$ is a BIBD.
 \end{prop}
 
 The remainder of this section will focus on $\bgw$ matrices and will closely follow Chapter 10 of \cite{combinatorics-of-symmetric-designs}. We first present a few simple constructions.
 
 \begin{prop}\label{gh proposition}
  Let $\gf(q) = \{a_0, \dots, a_{q-1}\}$, and define $H$ of order $q$ by $H_{ij} = a_ia_j$. Then $H$ is a $\gh(q,1)$.
 \end{prop}
 
 \begin{proof}
  Let $H$ be so defined, and let $i,j \in \{0,\dots,q-1\}$, $i \neq j$. Observe $\ssum_k (a_ia_k - a_ja_k) = (a_i - a_j)\ssum_k a_k$, hence each group element appears precisely once in $\{H_{ik}H_{jk}^{-1} : 0 \leq k < q\}$.
 \end{proof}

 \begin{ex}
  The GH$(4,1)$ (\ref{gh-ex}) was constructed using Proposition \ref{gh proposition}.
 \end{ex}
 
 \begin{prop}\label{gc proposition}
  Let $\gf(q) = \{a_0, \dots, a_{q-1}\}$, and define $W$ of order $q+1$ by
  \begin{defenum}
  \item $W_{ij} = \begin{cases}
            0 & \text{if } i = j = 0; \\
            1 & \text{if } i = 0 \text{ or } j = 0 \text{, but } i \neq j \text{; and} \\
            a_{i-1} - a_{j-1} & \text{otherwise.}
           \end{cases}$
  \end{defenum}
  Then $W$ is a $\bgw(q+1,q,q-1; \gf(q)^*)$.
 \end{prop}
 
 \begin{proof}
  Let $i,j \in \{1, \dots, q\}$, $i \neq j$. Then, for $\ell \neq j$,
  \[
  W_{i\ell}W_{j\ell}^{-1} = \frac{a_{i-1} - a_{\ell-1}}{a_{j-1} - a_{\ell-1}} = \frac{a_{i-1} - a_{j-1}}{a_{j-1} - a_{\ell-1}} + 1.
  \]
  As $\ell$ ranges over $\{1,\dots,q\}\bb\{j\}$, the difference $a_{j-1} - a_{\ell-1}$ ranges over $\gf(q)^*$. Since $W_{i0} = W_{j0} = 1$, the multiset $\{W_{i\ell}W_{j\ell}^{-1} : W_{i\ell} \neq 0 \neq W_{j\ell} \text{ and } 0 \leq \ell < q+1\}$ contains each element of $\gf(q)^*$ once. The remaining cases in which $i = 0$ or $j = 0$ are trivial.
 \end{proof}
 
 \begin{ex}
  A $\bgw(8,7,6; \gf(7)^*)$ formed from Proposition \ref{gc proposition} is given by
  \begin{equation}
  \arraycolsep=2.0pt\def\arraystretch{0.5}
  \left(
  \begin{array}{cccccccc}
0&6&6&6&6&6&6&6\\
3&0&1&2&3&4&5&6\\
3&4&0&3&1&6&2&5\\
3&5&6&0&4&2&1&3\\
3&6&4&1&0&5&3&2\\
3&1&3&5&2&0&6&4\\
3&2&5&4&6&3&0&1\\
3&3&2&6&5&1&4&0\\
\end{array}
\right),
  \end{equation}
  where the nonzero elements are the logarithms of some generator of $\gf(7)^*$. One can see that the matrix is skew-symmetric.
 \end{ex}
 
 We present a construction due to \cite{jungnickel-tonchev-perfect-codes-2}, which we do not prove here, that yields what are called the classical family of $\bgw$s; more than that, however, the matrices so constructed are what is termed $\omega$-circulant, a simple generalization of cirulant matrices.
 
 \begin{defin}\index{$\omega$-circulant}
 Let $G = \langle \omega \rangle$ be a finite cyclic group, and let $W$ be a matrix over $\Z[G]$ with first row $(\alpha_0, \dots, \alpha_{n-1})$. $W$ is $\omega$-circulant if $W_{ij} = \alpha_{j-i}$ if $i \leq j$ and $W_{ij} = \omega\alpha_{j-i}$ if $i > j$, where the indices are calculated modulo $n$.
 \end{defin}
 
 Finally, we are ready to present this simple and elegant construction. 
 
 \begin{prop}\label{trace-construction}
  Let $q$ be a prime power, and let $\beta$ be a primitive element of the extension of order $d$ of the field $\gf(q)$. Further, take $m = (q^d-1)/(q-1)$, and define $\omega = \beta^{-m} \in \gf(q)$, i.e. the norm of $\beta$. Finally, we claim that the $\omega$-circulant matrix with first row $(\tr(\beta^k))_{k=0}^{m-1}$ is a $\bgw(m,q^{d-1},q^{d-1}-q^{d-2}; \gf(q)^*)$.\Bnote{norm-func}
 \end{prop}
 
 \begin{ex}
  The matrix
  \begin{equation}
   \arraycolsep=2.0pt\def\arraystretch{0.5}
   \left(
   \begin{array}{ccccccccccccc}
0&2&2&2&0&1&2&2&1&2&0&2&0\\
0&0&2&2&2&0&1&2&2&1&2&0&2\\
1&0&0&2&2&2&0&1&2&2&1&2&0\\
0&1&0&0&2&2&2&0&1&2&2&1&2\\
1&0&1&0&0&2&2&2&0&1&2&2&1\\
2&1&0&1&0&0&2&2&2&0&1&2&2\\
1&2&1&0&1&0&0&2&2&2&0&1&2\\
1&1&2&1&0&1&0&0&2&2&2&0&1\\
2&1&1&2&1&0&1&0&0&2&2&2&0\\
0&2&1&1&2&1&0&1&0&0&2&2&2\\
1&0&2&1&1&2&1&0&1&0&0&2&2\\
1&1&0&2&1&1&2&1&0&1&0&0&2\\
1&1&1&0&2&1&1&2&1&0&1&0&0
   \end{array}
   \right)
  \end{equation}
is an $\omega$-circulant $\bgw(13,9,6; \gf(4)^*)$, where the nonzero elements are the logarithms of some generator of $\gf(4)^*$.
 \end{ex}

 We present a few brief remarks on the conjugate transpose and similar operations on $\bgw$s.
 
 \begin{prop}\label{bgw-conj-trans-prop}
  If $W$ is a $\bgw(v,k,\lambda;G)$, then $W^*$ is also a $\bgw(v,k,\lambda;G)$.
 \end{prop}

 \begin{proof}
  Consider $W$ as a matrix over the ring $\Q[G]$ so that it satisfies (\ref{gbrd-eq}) over $\Q[G]$. Let $\pi: \Q[G] \rightarrow \Q [G]/\Q G$ be the natural ring epimorphism. If $\pi W$ denotes the matrix $(\pi(W_{ij}))$, then (\ref{gbrd-eq}) becomes $(\pi W)(\pi W^*) = kI_v$, and hence $(\pi W)^{-1} = k^{-1}(\pi W^*)$. Therefore, $(\pi W^*)(\pi W) = kI_v$ so that $W^*W = kI_v + A$ for some $A = (a_{ij}\ssum_{g \in G}g)$ over the ideal $\Q G$. Moreover, since $\check W^t$ is a $\bibd$, there exist integers $a_g$ such that, for $i \neq j$, $\ssum_k w_{ki}w_{kj}^{-1} = \ssum_{g \in G}a_gg$ where $\ssum_{g \in G}a_g = \lambda$. Evidently, then, $A = \frac{\lambda}{|G|}(\ssum_{g \in G}g)(J_v - I_v)$, and the result follows.
 \end{proof}
 
 \begin{cor}
  If $W$ is a $\bgw(v,k,\lambda;G)$ where $G$ is abelian, then $\bar W$ and $W^t$ are also $\bgw(v,k,\lambda;G)$s.
 \end{cor}

 \begin{proof}
  Since the group is abelian, the map $g \mapsto g^{-1}$ is an automorphism; hence, by Proposition \ref{gbrd epimorphism}, $\bar W$ is also a $\bgw(v,k,\lambda;G)$. Then, by the proposition, $(\bar W)^* = W^t$ is also a $\bgw(v,k,\lambda;G)$.
 \end{proof}
 
 \dinkus
 
 % subsection %%%%%%%%%%%%%%%%%%%%%%%%%%%%%%%%%%%%%%%%%%%%%%%%%%%%%%%%%%%%%%%%%%%%%%%%%%%%%%%%%%%%%%%%%%%%%%%%%%%%%
 \subsection{Difference Set Construction III}
 
 In $\S3.4$, we introduced relative difference sets, and \ref{nega-weighing2} then uses a difference set relative to a normal subgroup of order 2 in order to construct a weighing matrix. We saw how the classical relative difference sets with parameters \ref{aff-diff-parameters}, together with Proposition \ref{projection-prop}, whenever $q$ is odd, can be used to construct the so-called classical weighing matrices. It so happens that this construction is a special case of something more general \cite[see][Theorem 10.3.1]{combinatorics-of-symmetric-designs}.
 
 \begin{thm}\label{rel-diff-bgw}
  Let $R$ be an $\rds(m,n,k,\lambda)$ in a group $G$ relative to a normal subgroup $N$. Let $g_0, \dots, g_{m-1}$ be all of the distinct coset representatives of $N$. Let $W$ be a $(0,N)$-matrix of order $m$ such that, for each $h \in N$, $W_{ij}=\alpha$ if and only if $g_iN \cap Rg_j = \{g_i\alpha\}$. Then $W$ is a $\bgw(m,k,\lambda n; N)$.
 \end{thm}
 
 \begin{proof}
  To begin, we first need that $|g_i \cap Rg_j| \leq 1$, for each $i,j \in \{0,\dots.m-1\}$. To that end, let $a,b \in g_iN \cap Rg_j$; then $a=g_i\alpha=tg_j$ and $b=g_i\beta=ug_j$, for $t,u \in R$ and $\alpha,\beta \in N$. Then $t=g_i\alpha g_j^{-1}$ and $u=g_i\beta g_j^{-1}$, hence $tu^{-1} = g_i\alpha\beta^{-1} g_i^{-1}$. It follows that $tu^{-1} \in N$, as $N$ is normal; thus, $t=u$ so that $a=b$. 
  
  For each $j \in \{0,\dots,m-1\}$, we have
  \[
   k=|Rg_j|=\left|\bigcup_{i=0}^{m-1} g_iN \cap Rg_j\right|=\sum_{i=0}^{m-1} |g_iN \cap Rg_j|,
  \]
  so we obtain that every column of $W$ has $k$ nonzero entries.
  
  Now, let $i,h \in \{0,\dots,m-1\}$, with $ \neq h$, and let $\gamma \in N$. First, assume that $\gamma=W_{ij}\bar{W_{hj}}$ for some $j$. Then there are unique $t,u \in R$ such that $tg_j=g_iW_{ij}$ and $ug_j=g_hW_{hj}$. It follows that $tu^{-1}=g_i\gamma g_h^{-1}$.
  
  On the other hand, suppose that $g_i\gamma g_h^{-1}=tu^{-1}$, for some $t,u \in R$. Then there is a unique $j$ such that $u^{-1}g_h \in g_jN$, that is, $u^{-1}g_h=g_j\beta$, for some $\beta \in N$. Then $ug_j=g_h\beta^{-1}$ and $tg_j=tu^{-1}g_h\beta^{-1}=g_i\gamma\beta^{-1}$ with $\gamma\beta^{-1} \in N$. Thus, $W_{ij}=\gamma\beta^{-1}$, $W_{hj}=\beta^{-1}$, and $\gamma=w_{ij}\bar{W_{hj}}$.
  
  It follows that the number of indices $j$ such that $\gamma=W_{ij}\bar{W_{hj}}$ is equal to the number of ordered pairs $(t,u)$ of elements of $R$ such that $tu^{-1}=g_i\gamma g_h^{-1}$. Since $g_i\gamma g_h^{-1} \not\in N$, there are exactly $\lambda$ such pairs. Therefore, $W$ is a $\bgw(m,k,\lambda n;N)$, and the proof is complete.
 \end{proof}
 
 Using the Theorem and the relative difference sets \ref{p-group-diff}, yields the following.
 
 \begin{cor}
  Let $G$ be an arbitrary group of order $p^n$. Then there is a $\gh(G;p^{2m-n})$, for every $m \geq n$.
 \end{cor}
 
 \begin{ex}
  Using the relative difference sets \ref{p-group-diff}, we construct a $\gh(C_4,4)$ over $C_4 \simeq \left\langle \left(\begin{smallmatrix} 0&+\\-&0 \end{smallmatrix}\right) \right\rangle$ given by
  \begin{equation}
  \begin{tiny}
  \arraycolsep=1.25pt\def\arraystretch{0.625}
   \left(\begin{array}{cccccccccccccccccccccccccccccccc}
    +&0&-&0&0&+&+&0&0&-&-&0&-&0&-&0&+&0&-&0&0&-&-&0&0&+&0&+&0&+&0&-\\
0&+&0&-&-&0&0&+&+&0&0&-&0&-&0&-&0&+&0&-&+&0&0&-&-&0&-&0&-&0&+&0\\
-&0&+&0&-&0&0&+&-&0&0&-&-&0&0&+&+&0&0&+&0&-&-&0&-&0&0&-&+&0&0&+\\
0&-&0&+&0&-&-&0&0&-&+&0&0&-&-&0&0&+&-&0&+&0&0&-&0&-&+&0&0&+&-&0\\
0&-&+&0&+&0&-&0&-&0&-&0&-&0&0&-&-&0&-&0&-&0&0&-&0&+&0&-&-&0&+&0\\
+&0&0&+&0&+&0&-&0&-&0&-&0&-&+&0&0&-&0&-&0&-&+&0&-&0&+&0&0&-&0&+\\
-&0&0&-&-&0&+&0&0&-&0&+&0&-&+&0&+&0&-&0&-&0&+&0&+&0&0&-&-&0&-&0\\
0&-&+&0&0&-&0&+&+&0&-&0&+&0&0&+&0&+&0&-&0&-&0&+&0&+&+&0&0&-&0&-\\
0&+&+&0&-&0&0&-&+&0&0&+&0&+&+&0&-&0&-&0&+&0&-&0&0&+&+&0&0&+&0&+\\
-&0&0&+&0&-&+&0&0&+&-&0&-&0&0&+&0&-&0&-&0&+&0&-&-&0&0&+&-&0&-&0\\
-&0&0&-&+&0&0&-&0&-&+&0&-&0&0&+&-&0&-&0&+&0&0&+&-&0&-&0&0&-&0&-\\
0&-&+&0&0&+&+&0&+&0&0&+&0&-&-&0&0&-&0&-&0&+&-&0&0&-&0&-&+&0&+&0\\
-&0&-&0&+&0&0&+&0&-&-&0&+&0&+&0&0&-&0&+&+&0&-&0&0&-&+&0&-&0&+&0\\
0&-&0&-&0&+&-&0&+&0&0&-&0&+&0&+&+&0&-&0&0&+&0&-&+&0&0&+&0&-&0&+\\
+&0&0&-&0&-&+&0&+&0&0&-&-&0&+&0&-&0&0&+&-&0&0&-&0&-&+&0&+&0&0&-\\
0&+&+&0&+&0&0&+&0&+&+&0&0&-&0&+&0&-&-&0&0&-&+&0&+&0&0&+&0&+&+&0\\
+&0&+&0&+&0&-&0&+&0&-&0&0&-&+&0&+&0&0&+&0&+&0&+&-&0&-&0&0&+&-&0\\
0&+&0&+&0&+&0&-&0&+&0&-&+&0&0&+&0&+&-&0&-&0&-&0&0&-&0&-&-&0&0&-\\
+&0&0&-&-&0&-&0&-&0&+&0&0&-&0&+&0&-&+&0&0&+&-&0&0&+&+&0&-&0&0&-\\
0&+&+&0&0&-&0&-&0&-&0&+&+&0&-&0&+&0&0&+&-&0&0&-&-&0&0&+&0&-&+&0\\
0&+&0&+&-&0&-&0&+&0&-&0&-&0&-&0&0&-&0&+&+&0&+&0&+&0&0&-&0&-&0&-\\
-&0&-&0&0&-&0&-&0&+&0&-&0&-&0&-&+&0&-&0&0&+&0&+&0&+&+&0&+&0&+&0\\
-&0&-&0&0&+&-&0&+&0&0&+&-&0&0&+&0&+&+&0&-&0&+&0&-&0&+&0&0&+&+&0\\
0&-&0&-&-&0&0&-&0&+&-&0&0&-&-&0&-&0&0&+&0&-&0&+&0&-&0&+&-&0&0&+\\
0&+&-&0&0&-&-&0&0&-&-&0&0&-&0&+&-&0&0&-&-&0&-&0&+&0&-&0&+&0&0&+\\
-&0&0&-&+&0&0&-&+&0&0&-&+&0&-&0&0&-&+&0&0&-&0&-&0&+&0&-&0&+&-&0\\
0&+&0&-&0&+&0&+&-&0&-&0&+&0&-&0&-&0&-&0&0&+&+&0&-&0&+&0&+&0&-&0\\
-&0&+&0&-&0&-&0&0&-&0&-&0&+&0&-&0&-&0&-&-&0&0&+&0&-&0&+&0&+&0&-\\
0&-&-&0&-&0&-&0&0&+&0&+&+&0&+&0&0&-&-&0&0&-&0&-&-&0&-&0&+&0&0&-\\
+&0&0&-&0&-&0&-&-&0&-&0&0&+&0&+&+&0&0&-&+&0&+&0&0&-&0&-&0&+&+&0\\
0&-&0&+&-&0&+&0&0&-&0&-&+&0&0&+&-&0&0&+&0&+&+&0&0&+&-&0&0&+&+&0\\
+&0&-&0&0&-&0&+&+&0&+&0&0&+&-&0&0&-&-&0&-&0&0&+&-&0&0&-&-&0&0&+\\
   \end{array}\right)
   \end{tiny}
  \end{equation}
 \end{ex}
 
 \dinkus
 
 % subsection %%%%%%%%%%%%%%%%%%%%%%%%%%%%%%%%%%%%%%%%%%%%%%%%%%%%%%%%%%%%%%%%%%%%%%%%%%%%%%%%%%%%%%%%%%%%%%%%%%%%%
 \subsection{Monomial Equivalence}
 
 As before, we can impose an equivalence on the set of all $v \times b$ $(0,G)$-matrices, which will play an important part in what is to come. 
 
 \begin{defin}\label{monomial-equiv}\index{monomial equivalence}
  Two $v \times b$ $(0,G)$-matrices $A_1$ and $A_2$ are said to be {\it monomially equivalent} if there are monomial $(0,G)$-matrices $P$ and $Q$ of orders $v$ and $b$, respectively, such that 
  \begin{defenum}
   \item $PA_1Q=A_2$.\Bnote{monomial}
  \end{defenum}
 \end{defin}
 
 In order to extend \ref{binary-equiv} and \ref{ternary-equiv} to BGW matrices, we begin by altering somewhat Definition \ref{gbrd definition} as in Part V of \cite{handbook}.
 
 \begin{defin}\label{c-gbrd defintion}\index{c-GBRD}
  let $G$ be some finite group, and let $A$ be a $v \times b$ $(0,G)$-matrix. If $A$ has $k$ non-zero entries in every column, and if there is an element $c \in (\frac{\lambda}{|G|}G)\bb\{1\} \subset \Z[G]$ such that
  \begin{defenum}
   \item\label{c-gbrd-eq} $AA^* = rI_v + c(J_v - I_v)$,
  \end{defenum}
  then we say that $A$ is a $c$-$\gbrd(v,k,\lambda-1;G)$, or a $c$-$\gbrd(v,b,r,k,\lambda-1;G)$ if more precision required.
 \end{defin}
 
 We can now properly extend the idea of normality to BGW matrices.
 
 \begin{defin}\index{balanced generalized weighing matrix!normal form}
  A $\bgw(v,k,\lambda;G)$ is said to be in {\it normal form} if it has the form
  \begin{equation}
   \begin{pmatrix}
    \zz_{v-k} & A_1 \\ \jj_{k} & A_2
   \end{pmatrix}.
  \end{equation}
 \end{defin}
 
 Note that of necessity, $A_1$ is a $\gbrd(v-k,v-1,k,k-\lambda,\lambda;G)$ and $A_2$ is a $c$-$\gbrd(k,v-1,k-1,\lambda,\lambda-1;G)$. These have the parameters of the residual and derived designs of a square $\bibd(v,k,\lambda)$, hence we call them, respectively, a residual GBRD and a derived $c$-GBRD.
 
 \biblio
\end{document}
