\documentclass[../../../main]{subfiles}

\begin{document}
 % subsection %%%%%%%%%%%%%%%%%%%%%%%%%%%%%%%%%%%%%%%%%%%%%%%%%%%%%%%%%%%%%%%%%%%%%%%%%%%%%%%%%%%%%%%%%%%%%%%%
 \subsection{Definitions}
 
  We begin with the case of real indeterminants.
 
 \begin{defin}\label{real-od-def}\index{orthogonal design}
  Let $x_1, \dots, x_u$ be real, commuting indeterminants, and let $X$ be an $n \times n$ matrix with entries from $\{0,\pm x_1, \dots, \pm x_u\}$. We say that $X$ is an {\it orthogonal design} if
  \begin{defenum}
   \item $XX^t = \left( \ssum_i s_ix_i^2 \right)I_n$.
  \end{defenum}
  We say that the orthogonal design is of order $n$ and type $(s_1, \dots, s_u)$, and we write $X$ is an $\od(n;s_1, \dots, s_u)$. If $\ssum_i s_i=n$, then we say that the OD is {\it full}.
 \end{defin}
 
 \begin{ex}
 The following are examples of full ODs of types $(1,1),(1,1,1,1)$, and $(1,1,1,1,1,1,1,1)$, respectively,
  \begin{equation}
  \arraycolsep=2.0pt\def\arraystretch{0.5}
  %\arraycolsep=1.25pt\def\arraystretch{0.625}
   \left(\begin{array}{cc} 
   a&b\\\bar{b}&a 
   \end{array}\right), 
   \arraycolsep=2.0pt\def\arraystretch{0.5}
   %\arraycolsep=1.25pt\def\arraystretch{0.625}
   \left(\begin{array}{cccc} 
   a&b&c&d\\\bar{b}&a&\bar{d}&c\\\bar{c}&d&a&\bar{b}\\\bar{d}&\bar{c}&b&a 
   \end{array}\right),
   \arraycolsep=2.0pt\def\arraystretch{0.5}
   %\arraycolsep=1.25pt\def\arraystretch{0.625}
   \left(\begin{array}{cccccccc}
   a&b&c&d&e&f&g&h\\
   \bar b&a&d&\bar c&f&\bar e&\bar h&g\\
   \bar c& \bar d&a&b&g&h&\bar e&\bar f\\
   \bar d&c&\bar b&a&h&\bar g&f&\bar e\\
   \bar e&\bar f&\bar g&\bar h&a&b&c&d\\
   \bar f&e&\bar h&g&\bar b&a&\bar d&c\\
   \bar g&h&e&\bar f&\bar c&d&a&\bar b\\
   \bar h&\bar g&f&e&\bar d&\bar c&b&a
   \end{array}\right),
  \end{equation}
  where we have used $\bar{x}$ to stand for $-x$. We do not have space to pursue these remarkable matrices here, but the interested reader is refered to \cite{seberry-od-2017} and the references cited therein for details.
 \end{ex}
 
 We now extend Definition \ref{real-od-def} to the case of complex and quaternary numbers.
 
 \begin{defin}\index{complex orthogonal design}\index{quaternary orthogonal design}
  Let $z_1, \dots, z_u$ be complex (resp. quaternary), commuting indeterminants, and let $X$ be a matrix of order $n$ with entries from $\{0,\varepsilon_1z_1, \dots, \varepsilon_uz_u\}$ and $\{\varepsilon_1z_1^*, \dots, \varepsilon_uz_u^*\}$, where each $\varepsilon_t \in \{\pm 1, \pm i\}$ (resp. $\varepsilon_t \in \{\pm 1, \pm i, \pm j, \pm k\}$). In the event that
  \begin{defenum}
   \item\label{qod-od-def} $XX^* = \left( \ssum_is_i|z_i|^2 \right)I_n$,
  \end{defenum}
  then we say that $X$ is a {\it complex} (resp. {\it quaternary}) orthogonal design of type $(s_1,\dots,s_u)$. We write $X$ is a $\cod(n;s_1,\dots,s_u)$ (resp. $\qod(n;s_1,\dots,s_u)$).
 \end{defin}
 
 \begin{ex}
  The matrix
  \begin{equation}
  \arraycolsep=1.25pt\def\arraystretch{0.625}
   \left(\begin{array}{cc}
    ia&b\\b&ia
   \end{array}\right),
  \end{equation}
  where $a$ and $b$ are real commuting indeterminants, is a $\cod(2;1,1)$.
 \end{ex}
 
 \begin{ex}
  The matrix
  \begin{equation}
  \arraycolsep=1.25pt\def\arraystretch{0.625}
   \left(\begin{array}{cccc}
    \bar a&\bar b&ic&ic\\
    b&\bar a&ic&i\bar c\\
    j\bar c&j\bar c&ka&kb\\
    j\bar c&jc&k\bar b&ka
   \end{array}\right),
  \end{equation}
  where $a,b$, and $c$ are real indeterminants, is a $\qod(4;1,1,2)$.
 \end{ex}
 
 \dinkus
 
 % subsection %%%%%%%%%%%%%%%%%%%%%%%%%%%%%%%%%%%%%%%%%%%%%%%%%%%%%%%%%%%%%%%%%%%%%%%%%%%%%%%%%%%%%%%%%%%%%%%%
 \subsection{Sequences and Circulants}
 
 Let $A$ be an $n \times n$ matrix with first row $(a_0,\dots,a_{n-1})$. Recall that $A$ is {\it circulant}\index{circulant matrix} if $A_{ij}=a_{j-i}$, where the indices are calculated modulo $n$. In this way, the entire matrix is determined by its first row; moreover, if $A$ and $B$ are two circulants of the same dimension, then $A^t,A+B$, and $AB$ are also circulant matrices. Therefore, to effect a study of circulant matrices, we can profitably study sequences. 
 
 What we are particularly interested with here is the following.
 
 \begin{defin}\index{circulant matrix!complementary}
  Let $\A=\{A_i\}$ be a finite collection of circulant matrices of the same dimension over a commutative ring $R$ endowed with an involution $\cdot^*$. The collection $\A$ is said to be {\it complementary} if
  \begin{defenum}
   \item $\ssum_i A_iA_i^* = aI$, for some $a \in R$,
  \end{defenum}
  where, as usual, $(m_{ij})^*=(m_{ji}^*)$.
 \end{defin}
 
 Note, however, that we can state this in terms of sequences. First, a definition.
 
 \begin{defin}\index{periodic autocorrelation}\index{aperiodic autocorrelation}\index{complementary sequences}
  Let $a_0=(a_{0,0}, \dots, a_{0,n-1})$ be a sequence in a commutative ring $R$ with the involution $\cdot^*$. The {\it $j$-th aperiodic} and {\it $j$-th periodic autocorrelations} of the sequence $a$ are given respectively by
  \begin{defenum}
   \item $N_j(a)=\ssum_{i=0}^{n-j-1}a_{0,i}a_{0,i+j}^*$, and
   \item $P_j(a)=\ssum_{i=0}^{n-1}a_{0,i}a_{0,i+j}^*$, indices calculated modulo $n$.
  \end{defenum}
  If $a_1=(a_{1,0},\dots,a_{1,n-1}),\dots,a_m=(a_{m,1},\dots,a_{m,n-1})$ are any other sequence in $R$, then $a_0,\dots,a_m$ are {\it complementary}\index{complementary sequences} if
  \begin{defenum}[resume]
   \item\label{periodic-comp} $\ssum_i P_j(a_i)=0$, for every $j \in \{1, \dots, n-1\}$.
  \end{defenum}
  For the case in which $m=1$, we say that we have a {\it Golay pair}.
 \end{defin}
 
 We see immediately that $P_j(a)=N_j(a) + N_{n-j}(a)^*$; hence, if $N_j(a)+N_j(b)=0$, for every $j \in \{1,\dots,n-1\}$, then $a$ and $b$ are complementary. However, vanishing periodic autocorrelations does not in general imply vanishing aperiodic autocorrelations.
 
 The importance of the periodic correlation is given by the fact that if the first row of the circulant $A$ continues to be $a=(a_0,\dots,a_{n-1})$, then the first row of $AA^*$ is given by $(\ssum_i|a_i|^2,P_{n-1}(a), \dots,P_1(a))$. So we see that complementary circulants and complementary sequences are one and the same.
 
 Complementary sequences and orthogonal designs are connected in an intimate way; in fact, complementary sequences offer many elegant constructions of ODs. To make the connection precise, we need to allow sequence elements to be indeterminants whether real, complex, or quaternary, and where the involution is taken to be conjugation.
 
 \begin{prop}\label{2-circs}
  Let $\{z_1,\dots,z_u\}$ be commuting quaternary indeterminants, and let $a=(a_0,\dots,a_{n-1})$ and $b=(b_0,\dots,b_{n-1})$ be complementary sequences with entries from $\{0,\varepsilon_0z_0, \dots, \varepsilon_uz_u\}$, where $\varepsilon_t \in \{\pm 1, \pm i, \pm j, \pm k\}$, such that $\ssum_i(|a_i|^2 + |b_i|^2)=\ssum_is_ix_i^2$. Then
  \begin{equation}\label{plug-in-matrix}
  \arraycolsep=2.0pt\def\arraystretch{0.5}
  %\arraycolsep=1.25pt\def\arraystretch{0.625}
   \left(\begin{array}{cc}
    A&B\\-B^*&A^*
   \end{array}\right)
  \end{equation}
  is a $\qod(2n;s_1,\dots,s_u)$, where $A$ and $B$ are the circulants with first rows $a$ and $b$, respectively.
 \end{prop}
 
 \begin{proof}
  A restatement of \ref{periodic-comp}.
 \end{proof}
 
 The matrix (\ref{plug-in-matrix}) will feature as a submatrix in our later work where the sequences will be composed of matrices. We will need one further idea before we proceed. 

 \begin{defin}\index{cross-correlation}
  Let $a=(a_0,\dots,a_{n-1})$ and $b=(b_0,\dots,b_{n-1})$ be two sequences over a ring with involution $\cdot^*$. The {\it $j$-th cross-correlation} of $a$ by $b$ is given by
  \begin{defenum}
   \item $C_j(a,b)=\ssum_{i=0}^{n-1}a_ib_{i+j}^*$, for each $j \in \{1,\dots,n-1\}$.
  \end{defenum}
 \end{defin}
 
 If $A$ has first row $a=(a_0,\dots,a_{n-1})$, and if $B$ has first row $b=(b_0,\dots,b_{n-1})$, then the first row of the circulant $AB^*$ is $(\ssum_ia_ib_i^*,C_{n-1}(a,b),\dots,C_1(a,b))$. Note, however, that $C_j(a,b)$ is not in general equal to $C_j(b,a)$. So, care must be taken in extending Proposition \ref{2-circs} since amicability is required in maintaining orthogonality when substituting into an OD.
 
 Much more can be said about this most useful topic. The interested reader is may consult \cite{seberry-od-2017} and \cite{seberry-yamada-hmatrices-designs} for a wealth of material.
 
 \biblio
\end{document}
