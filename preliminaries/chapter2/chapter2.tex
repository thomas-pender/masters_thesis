\documentclass[../../main]{subfiles}

\newendnotes{b}
\input{ch2notes.txt}
\setcounter{bnote}{\value{anote}}
\newcommand{\Bnote}[1]{\bnote{#1}\big)}

\begin{document}

This chapter focuses on a third combinatorial configuration, namely, weighing matrices, and their generalizations. We begin with a brief look at weighing matrices themselves, before moving onto to consider two generalizations. The first generalization is allowing the entries of a weighing matrix to be chosen from a finite group. This idea is then synthesized with the ideas of $\S1$. Finally, we allow the entries of a weighing matrix to be taken from sets of indeterminants and consider the utility of such an approach.
 
 \section{\centering Weighing Matrices}
 
 In this section, we study weighing matrices. Apart from having numerous applications, these objects are of significant theoretical interest. Though we will be focusing on certain generalizations of these objects, it is worthwhile to consider their simplest case.
 
 \dinkus
 
 \subfile{./weighing/weighing}
 
 \section{\centering Balanced Generalized Weighing Matrices}
 
 We come now to the central object of our study, namely, the so-called balanced generalized weighing matrices. These objects are intriguing in their own right as we shall see, but they also have interesting applications in the construction of other configurations.
 
 \dinkus
 
 \subfile{./bgw/bgw}
 
 \section{\centering Orthogonal Designs}
 
 In the previous section, we covered one generalization of weighing matrices, namely, we allowed the nonzero enetries to come from a finite group. In this section, we pursue another generalization in a different direction. In particular, instead of allowing the nonzero entries to come from some group, we will take the nonzero entries to be indeterminants---real, complex, or quaternary\Bnote{quaternary}.
 
 \dinkus
 
 \subfile{./orthogonaldesigns/orthogonaldesigns}
 
 \singlespace
 
 \addcontentsline{toc}{section}{Notes}
 \section*{\centering Notes}
 \thebnotes
 
 \doublespacing
 
 \biblio
\end{document}
