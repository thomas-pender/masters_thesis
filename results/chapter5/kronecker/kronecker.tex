\documentclass[../../../main]{subfiles}

\begin{document}

% subsection %%%%%%%%%%%%%%%%%%%%%%%%%%%%%%%%%%%%%%%%%%%%%%%%%%%%%%%%%%%%%%%%%%%%%%%%%%%%%%%%
\subsection{Definitions}

 Let $A$ be an $n \times m$ matrix, and let $B$ be an $s \times t$ matrix, each with entries from some field. The {\it Kronecker product}\index{Kronecker product} of $A$ by $B$ is the $ns \times mt$ matrix given by $A \otimes B = (A_{ij}B)$. This noncommutative matrix product has the following immediate properties wherever the sizes of the matrices make sense.
 
 \begin{lem}
  Let $A,B,C$ and $D$ be matrices over some field $F$, and let $\lambda \in F$. Then:
  \begin{defenum}
   \item $\lambda A \otimes B = A \otimes \lambda B = \lambda(A \otimes B)$,
   \item $A \otimes B = (A \times I)(I \otimes B) = (I \otimes B)(A \otimes I)$,
   \item $(A+B) \otimes C = (A \otimes C) + (B \otimes C)$,
   \item $A \otimes (B+C) = (A \otimes B) + (A \otimes C)$,
   \item $(A \otimes B)(C \otimes D) = AC \otimes BD$,
   \item $(A \otimes B)^t = A^t \otimes B^t$, and
   \item if $A$ is $n \times n$ and $B$ is $m \times m$, then $\ddet(A \otimes B)=[\ddet(A)]^m[\ddet(B)]^n$.
  \end{defenum}
 \end{lem}
 
 There are a number of immediate applications of the Kronecker product to what we have done so far. For instance, if $H$ is a $\w(n_1,k_1)$ and $K$ a $\w(n_2,k_2)$, then $H \otimes K$ is a $\w(n_1n_2,k_1k_2)$.
 
 \begin{ex}\label{regular-ex}
  Let $H$ be the Hadamard matrix of order 4 given by
  \begin{defenum}
   \item $
   \arraycolsep=1.25pt\def\arraystretch{0.625}
   \left(\begin{array}{cccc}
    -&+&+&+\\
    +&-&+&+\\
    +&+&-&+\\
    +&+&+&-
   \end{array}\right).
   $
  \end{defenum}
  Then $H \otimes H$ is the Hadamard matrix of 16 given by
  \begin{defenum}[resume]
   \item $
   \arraycolsep=1.25pt\def\arraystretch{0.625}
   \left(\begin{array}{cccccccccccccccc}
+&-&-&-&-&+&+&+&-&+&+&+&-&+&+&+\\
-&+&-&-&+&-&+&+&+&-&+&+&+&-&+&+\\
-&-&+&-&+&+&-&+&+&+&-&+&+&+&-&+\\
-&-&-&+&+&+&+&-&+&+&+&-&+&+&+&-\\
-&+&+&+&+&-&-&-&-&+&+&+&-&+&+&+\\
+&-&+&+&-&+&-&-&+&-&+&+&+&-&+&+\\
+&+&-&+&-&-&+&-&+&+&-&+&+&+&-&+\\
+&+&+&-&-&-&-&+&+&+&+&-&+&+&+&-\\
-&+&+&+&-&+&+&+&+&-&-&-&-&+&+&+\\
+&-&+&+&+&-&+&+&-&+&-&-&+&-&+&+\\
+&+&-&+&+&+&-&+&-&-&+&-&+&+&-&+\\
+&+&+&-&+&+&+&-&-&-&-&+&+&+&+&-\\
-&+&+&+&-&+&+&+&-&+&+&+&+&-&-&-\\
+&-&+&+&+&-&+&+&+&-&+&+&-&+&-&-\\
+&+&-&+&+&+&-&+&+&+&-&+&-&-&+&-\\
+&+&+&-&+&+&+&-&+&+&+&-&-&-&-&+
   \end{array}\right),
   $
  \end{defenum}
  where the block structure is evident\Enote{kronecker-note}.
 \end{ex}
 
 As one further application of the standard Kronecker product before we move on, we will consider regular Hadamard matrices.
 
 \begin{defin}\index{regular Hadamard matrix}
  A Hadamard matrix is {\it row regular} (or {\it column regular}) if it's rows (resp. columns) have a constant sum. A Hadamard matrix is {\it regular} if it is both row and column regular.
 \end{defin}
 
 It isn't difficult to see that a Hadamard matrix is row regular or column regular if and only if it is regular. Moreover, if the constant row (column) sum is $s$, then the order of the matrix is $s^2$ \cite[see][Chapter 4]{combinatorial-designs}. By the definition of the standard Kronecker product, we also see that if $H$ and $K$ are regular Hadamard matrices, then so is $H \otimes K$.
 
 Example \ref{regular-ex} has evinced the existence of a regular $\w(4,4)$. It is also known that there is a regular $\w(36,36)$ \cite[see][Chapter 4]{combinatorial-designs}. Also, if there is a Hadamard matrix of order $n$, then there is a symmetric, regular, Hadamard matrix with constant block diagonal of order $n^2$ \cite[see][Part V]{handbook}. We then have the following result.
 
 \begin{prop}
  If there exist Hadamard matrices of orders $n_1, \dots, n_k$, then there is a regular Hadamard matrix of order $4^a9^bn_1^2\cdots n_k^2$, for $a,b \in \Z_+$ and $a \geq b$.
 \end{prop}
 
 Having convinced ourselves of the utility of the standard Kronecker product, we generalize in the following way. Let $\MM$ be a collection of $n \times m$ matrices with entries from some commutative ring $R$, and let $\Xi$ be an $s \times t$ matrix over the collection of endofunctions of $\MM$. For each $A \in \MM$, the $ns \times mt$ matrix $\Xi \otimes A$ over $R$ is defined as $(\Xi_{ij}(A))$. If certain properties of $A$ are left invariant under the elements of $\Xi$, then these properties will be reflected in the block structure of $\Xi \otimes A$.
 
 With the above ideas in mind, if we have a collection $\MM$ of objects, we desire a subset of the collection of endofunctions of $\MM$ that are property preserving. If $\MM$ is a collection of incidence structures, then a set of bijections of $\MM$ that preserve incidence is called a {\it group of symmetries}\index{group of symmetries} of $\MM$. If $\Xi$ is a BGW over a group of symmetries, then $\Xi \otimes A$ has both inter-block regularites and intra-block regularities.
 
 \dinkus
 
 % subsection %%%%%%%%%%%%%%%%%%%%%%%%%%%%%%%%%%%%%%%%%%%%%%%%%%%%%%%%%%%%%%%%%%%%%%%%%%%%%%%%
\subsection{A First Application: Block designs}

In the previous subsection, we began a brief discussion of groups of symmetries of incidence structures. As an example, let $\D=(X,\B)$ be a $\bibd(v,b,r,k,\lambda)$, and let $\B = \bigcup_{i=1}^m \B_i$ be a partition of the blocks of $\D$. If $G_i$ acts on $\B_i$, then, $(X,\bigcup_{i=1}^m\B_ig_i)$, for $g_i \in G_i$, is again a $\bibd(v,b,r,k,\lambda)$. Let $A=A(\D)$ be the incidence matrix of $\D$, and assume that $A = \left(\begin{smallmatrix} A(X,\B_1) & \cdots & A(X,\B_m) \end{smallmatrix}\right)$. Then $G=\pprod_iG_i$ acts naturally on the columns of $A$. If in addition $G_i$ is sharpley transitive on $\B_i$, and if each point $x \in X$ appears $r_i$ times in $\B_i$, then 
 \[
 \arraycolsep=1.25pt\def\arraystretch{0.625}
 \abs{G}^{-1}\sum_{g \in G}Ag = (\begin{array}{ccc} r_1b_1^{-1}J_{v \times b_1} & \cdots & r_mb_m^{-1}J_{v \times b_m} \end{array}),
 \]
 where $b_i=\abs{\B_i}$. Therefore, if $r_ib_j=r_jb_i$, for all $i,j \in \{1,\dots,m\}$, then $\ssum_{g \in G}Ag$ is an integer multiple of $J_{v \times b}$ since $b_i \mid |G|$. Since $G$ acts on the columns of $A$, we also have that $Xg(Yg)^t = XY^t$, for each $X,Y \in AG$. 
 
 This motivates the following. If $G$ is a group of symmetries of a collection $\MM$ of $\bibd(v,b,r,k,\lambda)$s such that $\ssum_{g \in G}Xg = \alpha_XJ$, for some $\alpha_X \in \N$ and all $X \in \MM$; and if $Xg(Yg)^t = XY^t$, for all $X,Y \in \MM$; then we say that $G$ is an {\it admissible} group of symmetries of $\D$\index{admissible group of symmetries}. We then have the following.
 
 \begin{thm}\label{bibd-bgw-thm}
  Let $\MM$ be a collection of $\bibd(v,b,r,k,\lambda)$s, and let $G$ be an admissible group of symmetries of $\MM$. If $\Xi$ is a $\bgw(w,\ell,\mu;G)$ such that $kr\mu=v\lambda\ell$, then $\Xi \otimes X$ is a $\bibd(vw,bw,r\ell,k\ell,\lambda\ell)$, for any $X \in \MM$.
 \end{thm}
 
 \begin{proof}
  Straightforward calculation. See Theorem 2.4 of \cite{ionin-bgw-bibd} for details.
 \end{proof}
 
 \begin{cor}
  If $X$ is quasi-residual, then so is $\Xi \otimes X$.
 \end{cor}
 
 \begin{ex}
  Let $q=p^n$ be a prime power, and let $H$ be a $\gh(q,1)$ over $\text{EA}(q)$ where the elements have the usual representation of $(0,1)$-matrices. Then it isn't difficult to see that $A=(\begin{smallmatrix} I_q \otimes \jj_q & H \end{smallmatrix})$ is a $\bibd(q^2,q+q^2,1+q,q,1)$. Importantly, the columns are placed into $1+q$ consecutive disjoint groups of $q$ columns each such that each point appears precisely once in every group\Enote{resolvable-note}. Let $G$ be the group which cyclically permutes the blocks in each group so that $|G|=q$.
  
  If $1+q$ is a prime power, then there is a $\bgw(2+q,1+q,q;C_q)$, say $\Xi$, where the group elements cyclically permute the blocks of each parition class in the above design. The parametric conditions of the theorem are met, hence $\Xi \otimes A$ is a $\bibd(q^2(2+q),(q+q^2)(2+q),(1+q)^2,q(1+q),1+q)$. More generally, we can take $\Xi$ to be any $\bgw((p^{n+1}-1)/(p-1),p^n,p^n-p^{n-1};C_q)$, where $p=1+q$.
 \end{ex}

\dinkus

% subsection %%%%%%%%%%%%%%%%%%%%%%%%%%%%%%%%%%%%%%%%%%%%%%%%%%%%%%%%%%%%%%%%%%%%%%%%%%%%%%%%
\subsection{A Second Application: Bkaskar Rao designs}

It isn't difficult to extend the ideas of Theorem \ref{bibd-bgw-thm} to the more general Bhaskar Rao designs. In order to accomplish this, however, we need to extend the idea of admissible groups of symmetries.

Let $\MM$ be a collection of $\gbrd(v,b,r,k,\lambda;H)$s. Then a group of bijections $G$ of $\MM$ is an admissible group of symmetries\index{admissible group of symmetries} if 
\begin{enumerate*}[(a)]
\item $\sum_{g \in G}Xg=\alpha_XHJ$, for some $\alpha_X \in \N$ and every $X \in \MM$, and 
\item $Xg(Yg)^* = XY^*$, for every $X,Y \in \MM$.
\end{enumerate*}

The following is then a simple generalization of Theorem \ref{bibd-bgw-thm}

\begin{thm}\label{gbrd-bgw-thm}
 Let $\MM$ be a collection of $\gbrd(v,b,r,k,\lambda;H)$s, and let $G$ be an admissible group of symmetries of $\MM$. If $\Xi$ is a $\bgw(w,\ell,\mu;G)$ such that $kr\mu=v\lambda\ell$, then $\Xi \otimes X$ is a $\gbrd(vw,bw,r\ell,k\ell,\lambda\ell;H)$, for any $X \in \MM$.
\end{thm}

\begin{ex}
 The following example was noted in \cite{pender_2020}. The following $\bgw(15,7,3;C_3)$ was found by computational means in \cite{mathon-signings}
 \begin{defenum}
  \item\label{bgw.15.7.3.3} $
  \arraycolsep=2.0pt\def\arraystretch{0.5}
  \left(\begin{array}{ccccccccccccccc}
0&3&3&3&3&3&3&3&0&0&0&0&0&0&0\\
0&3&2&0&1&0&0&0&0&0&2&0&2&2&3\\
0&0&3&2&0&1&0&0&3&0&0&2&0&2&2\\
0&0&0&3&2&0&1&0&2&3&0&0&2&0&2\\
0&0&0&0&3&2&0&1&2&2&3&0&0&2&0\\
0&1&0&0&0&3&2&0&0&2&2&3&0&0&2\\
0&0&1&0&0&0&3&2&2&0&2&2&3&0&0\\
0&2&0&1&0&0&0&3&0&2&0&2&2&3&0\\
3&3&1&0&2&0&0&0&3&2&0&1&0&0&0\\
3&0&3&1&0&2&0&0&0&3&2&0&1&0&0\\
3&0&0&3&1&0&2&0&0&0&3&2&0&1&0\\
3&0&0&0&3&1&0&2&0&0&0&3&2&0&1\\
3&2&0&0&0&3&1&0&1&0&0&0&3&2&0\\
3&0&2&0&0&0&3&1&0&1&0&0&0&3&2\\
3&1&0&2&0&0&0&3&2&0&1&0&0&0&3\\
  \end{array}\right),
   $
 \end{defenum}
 where the nonzero entries are logarithms of some generator of $C_3$. Note that the core of \ref{bgw.15.7.3.3} is composed of 4 circulant matrices. Taking $R$ to be the residual part of \ref{bgw.15.7.3.3}, and letting $g=\left(\begin{smallmatrix} O & I_7 \\ \omega I_7 & O \end{smallmatrix}\right)$, we see that $G=\sharps{g}$ forms an admissible group of symmetries for $R$. For any $n \in N$, there is a $\bgw((7^{n+1}-1)/6,7^n,6 \cdot 7^{n-1};G)$ which satisfy the parametric conditions of Theorem \ref{gbrd-bgw-thm}. Therefore, for any $n>0$, there is a $\gbrd(4(7^{n+1}-1)/3,2\cdot 7^{n+1},7^{n+1},4 \cdot 7^n,3\cdot 7^n; C_3)$.\Enote{bgw.15.7.3-comp}
\end{ex}

This concludes our introduction to the generalized Kronecker product.
 
\biblio
\end{document}
