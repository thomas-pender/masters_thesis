%\providecommand{\main}{../../main}
\documentclass[../../main]{subfiles}

\newendnotes{e}
\input{ch5notes.txt}
\setcounter{enote}{\value{dnote}}
% \newcommand{\Enote}[1]{\enote{#1}\big)}
\newcommand{\Enote}[1]{\enote{#1}}

\begin{document}

This chapter serves as a particular application of a more general construction
to be given in the following chapter. Here a new family of balanced weighing
matrices are constructed, and an equivalence to certain association schemes is
developed. The work shown here is a modified version of \cite{new-bw}. 

\fancyhf{}

\fancyhead[RO,LE]{\thepage}
\fancyhead[CO]{\S\thesection. Generalized Kronecker Product}
\fancyhead[CE]{Chapter \thechapter. A New Family of Balanced Weighing Matrices and Association Schemes}

 \section{\centering Generalized Kronecker Product}
 
 In this section, we briefly review the Kronecker product and a few of its properties and applications, namely, we will see its applications to regular Hadamard matrices. Following this, we will consider a simple generalization of the Kronecker product in order to allow particular bijections to act on a matrix.
 
 \dinkus
 
 \subfile{./kronecker/kronecker}
 
 \fancyhf{}

 \fancyhead[RO,LE]{\thepage}
 \fancyhead[CO]{\S\thesection. A New Family of Balanced Weighing Matrices}
 \fancyhead[CE]{Chapter \thechapter. A New Family of Balanced Weighing Matrices and Association Schemes}

 \section{\centering A New Family of Balanced Weighing Matrices}
 
 Here the generalized Kronecker product of the previous subsection will be put to use in constructing a new family of balanced weighing matrices. Additionally, the simplex codes appear again and are applied in the construction. The construction presented here is indicative of a general method to be presented in the following chapter.
 
 \dinkus
 
 \subfile{./bw/bw}
 
 \fancyhf{}

 \fancyhead[RO,LE]{\thepage}
 \fancyhead[CO]{\S\thesection. Weighing Matrices and Association Schemes}
 \fancyhead[CE]{Chapter \thechapter. A New Family of Balanced Weighing Matrices and Association Schemes}

 \section{\centering Weighing Matrices and Association Schemes}
 
 In \cite{distance-regular-graphs} it is shown that symmetric designs are equivalent to certain 3-class association schemes. It is a natural question to ask whether or not a balanced weighing matrix could be related to this scheme, particularly an augmentation. This is the goal of this subsection.
 
 \dinkus
 
 \subfile{./scheme/scheme}
 
 \singlespace
 
 \fancyhf{}

 \fancyhead[RO,LE]{\thepage}
 \fancyhead[CO]{Notes}
 \fancyhead[CE]{Chapter \thechapter. A New Family of Balanced Weighing Matrices and Association Schemes}

 \addcontentsline{toc}{section}{Notes}
 \section*{\centering Notes}
 \theenotes
 
 \doublespacing
 
 \biblio
\end{document}
