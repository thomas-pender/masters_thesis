\documentclass[../../../main]{subfiles}

\begin{document}

% subsection %%%%%%%%%%%%%%%%%%%%%%%%%%%%%%%%%%%%%%%%%%%%%%%%%%%%%%%%%%%%%%%%%%%%%%%%%%%%%%%%
\subsection{A First Application: Weighing matrices}

It is a peculiarity of the constuction that the application of the method to weighing matrices is predicated upon the weight of the matrix being a prime power. Indeed, as in the previous chapter, the generalized simplex codes are applied to the derived part of the matrix by substituting the rows of the derived part for the letters of the code. The simplex code, as the reader may remember, has a prime power number of letters.

To begin, let $W = \left(\begin{smallmatrix} \zz & R \\ \jj & D \end{smallmatrix}\right)$ be a $\w(v,q)$ weighing matrix in normal form, where $q$ is some prime power. Then it is easy to see that $RR^t = kI$ and $DD^t = kI-J$. We say that $W$ is the {\it seed matrix} of the construction. 

Let $\simplex_{q,n} = \ssum_{\alpha\in\gf(q)^*} \alpha A_\alpha$, and define $A_0 = J - \ssum_{\alpha\in\gf(q)^*} A_\alpha$. Index the rows of $D$ by the elements of $\gf(q)$, and take $\DD = \ssum_{\alpha\in\gf(q)} A_\alpha \otimes r_\alpha$. Next, by our previous work, there is a $\w((q^n-1)/(q-1),q^{n-1})$, say $H$, for every $n>1$. We define $\rr = H \otimes R$.

We claim that $\WW = \left(\begin{smallmatrix} \zz & \rr \\ \jj & \DD \end{smallmatrix}\right)$ is again a weighing matrix. Indeed, $\rr\rr^t = (H \otimes R)(H \otimes R)^t = HH^t \otimes RR^t = q^nI$. Further, by Lemma \ref{oa-lem},
\begin{align*}
\DD\DD^t &= \left( \sum_\alpha A_\alpha \otimes r_\alpha \right) \left( \sum_\alpha A_\alpha \otimes r_\alpha \right) \\
&= \sum_\alpha A_\alpha A_\alpha^t \otimes r_\alpha r_\alpha^t + \sum_{\alpha\neq\beta} A_\alpha A_\beta^t \otimes r_\alpha r_\beta^t \\
&= (q-1)\sum_\alpha A_\alpha A_\alpha^t - \sum_{\alpha\neq\beta} A_\alpha A_\beta^t \\
&= (q-1)\left( \frac{q^{n-1}-1}{q-1}J + q^{n-1}I \right) - q^{n-1}(J-I) \\
&= q^nI - J.
\end{align*}
Finally,
\begin{align*}
 \rr\DD^t &= (H \otimes R)\left( \sum_\alpha A_\alpha \otimes r_\alpha \right)^t \\
 &= \sum_\alpha HA_\alpha^t \otimes Rr_\alpha^t \\
 &= \sum_\alpha HA_\alpha^t \otimes O \\
 &= O.
\end{align*}
It follows that $\WW$ is a weighing matrix. We record this result below.

\begin{thm}
 If there exists a $\w(v,q)$ weighing matrix of odd prime power weight, then there is a weighing matrix with parameters
 \begin{equation}
  \left(
  \frac{(v-1)(q^n-1)}{q-1} + 1, q^n
  \right).
 \end{equation}
\end{thm}

\begin{ex}
 Take as a seed matrix the $\w(8,5)$ shown below
 \begin{defenum}
  \item $
  \arraycolsep=1.25pt\def\arraystretch{0.625}
  \left(\begin{array}{cccccccc}
0&+&0&0&-&-&+&+\\
0&0&+&0&-&+&-&+\\
0&0&0&+&-&+&+&-\\
+&+&+&+&+&0&0&0\\
+&+&-&-&0&+&0&0\\
+&-&+&-&0&0&+&0\\
+&-&-&+&0&0&0&+\\
+&0&0&0&-&-&-&-\\
  \end{array}\right).
  $
 \end{defenum}
 Applying the construction, we find a $\w(43,25)$
 \begin{defenum}[resume]
  \item \begin{tiny}$
  \arraycolsep=1.25pt\def\arraystretch{0.625}
  \left(\begin{array}{ccccccccccccccccccccccccccccccccccccccccccc}
0&0&0&0&0&0&0&0&+&0&0&-&-&+&+&+&0&0&-&-&+&+&+&0&0&-&-&+&+&+&0&0&-&-&+&+&+&0&0&-&-&+&+\\
0&0&0&0&0&0&0&0&0&+&0&-&+&-&+&0&+&0&-&+&-&+&0&+&0&-&+&-&+&0&+&0&-&+&-&+&0&+&0&-&+&-&+\\
0&0&0&0&0&0&0&0&0&0&+&-&+&+&-&0&0&+&-&+&+&-&0&0&+&-&+&+&-&0&0&+&-&+&+&-&0&0&+&-&+&+&-\\
0&+&0&0&-&-&+&+&0&0&0&0&0&0&0&-&0&0&+&+&-&-&+&0&0&-&-&+&+&-&0&0&+&+&-&-&+&0&0&-&-&+&+\\
0&0&+&0&-&+&-&+&0&0&0&0&0&0&0&0&-&0&+&-&+&-&0&+&0&-&+&-&+&0&-&0&+&-&+&-&0&+&0&-&+&-&+\\
0&0&0&+&-&+&+&-&0&0&0&0&0&0&0&0&0&-&+&-&-&+&0&0&+&-&+&+&-&0&0&-&+&-&-&+&0&0&+&-&+&+&-\\
0&+&0&0&-&-&+&+&-&0&0&+&+&-&-&0&0&0&0&0&0&0&-&0&0&+&+&-&-&+&0&0&-&-&+&+&+&0&0&-&-&+&+\\
0&0&+&0&-&+&-&+&0&-&0&+&-&+&-&0&0&0&0&0&0&0&0&-&0&+&-&+&-&0&+&0&-&+&-&+&0&+&0&-&+&-&+\\
0&0&0&+&-&+&+&-&0&0&-&+&-&-&+&0&0&0&0&0&0&0&0&0&-&+&-&-&+&0&0&+&-&+&+&-&0&0&+&-&+&+&-\\
0&+&0&0&-&-&+&+&+&0&0&-&-&+&+&-&0&0&+&+&-&-&0&0&0&0&0&0&0&+&0&0&-&-&+&+&-&0&0&+&+&-&-\\
0&0&+&0&-&+&-&+&0&+&0&-&+&-&+&0&-&0&+&-&+&-&0&0&0&0&0&0&0&0&+&0&-&+&-&+&0&-&0&+&-&+&-\\
0&0&0&+&-&+&+&-&0&0&+&-&+&+&-&0&0&-&+&-&-&+&0&0&0&0&0&0&0&0&0&+&-&+&+&-&0&0&-&+&-&-&+\\
0&+&0&0&-&-&+&+&-&0&0&+&+&-&-&+&0&0&-&-&+&+&+&0&0&-&-&+&+&0&0&0&0&0&0&0&-&0&0&+&+&-&-\\
0&0&+&0&-&+&-&+&0&-&0&+&-&+&-&0&+&0&-&+&-&+&0&+&0&-&+&-&+&0&0&0&0&0&0&0&0&-&0&+&-&+&-\\
0&0&0&+&-&+&+&-&0&0&-&+&-&-&+&0&0&+&-&+&+&-&0&0&+&-&+&+&-&0&0&0&0&0&0&0&0&0&-&+&-&-&+\\
0&+&0&0&-&-&+&+&+&0&0&-&-&+&+&+&0&0&-&-&+&+&-&0&0&+&+&-&-&-&0&0&+&+&-&-&0&0&0&0&0&0&0\\
0&0&+&0&-&+&-&+&0&+&0&-&+&-&+&0&+&0&-&+&-&+&0&-&0&+&-&+&-&0&-&0&+&-&+&-&0&0&0&0&0&0&0\\
0&0&0&+&-&+&+&-&0&0&+&-&+&+&-&0&0&+&-&+&+&-&0&0&-&+&-&-&+&0&0&-&+&-&-&+&0&0&0&0&0&0&0\\
+&+&-&-&0&+&0&0&+&-&-&0&+&0&0&-&+&-&0&0&+&0&0&0&0&-&-&-&-&+&+&+&+&0&0&0&-&-&+&0&0&0&+\\
+&-&+&-&0&0&+&0&+&-&-&0&+&0&0&0&0&0&-&-&-&-&+&+&+&+&0&0&0&-&-&+&0&0&0&+&-&+&-&0&0&+&0\\
+&-&-&+&0&0&0&+&+&-&-&0&+&0&0&+&+&+&+&0&0&0&-&-&+&0&0&0&+&-&+&-&0&0&+&0&0&0&0&-&-&-&-\\
+&0&0&0&-&-&-&-&+&-&-&0&+&0&0&-&-&+&0&0&0&+&-&+&-&0&0&+&0&0&0&0&-&-&-&-&+&+&+&+&0&0&0\\
+&+&+&+&+&0&0&0&+&-&-&0&+&0&0&+&-&-&0&+&0&0&+&-&-&0&+&0&0&+&-&-&0&+&0&0&+&-&-&0&+&0&0\\
+&+&-&-&0&+&0&0&-&+&-&0&0&+&0&0&0&0&-&-&-&-&-&-&+&0&0&0&+&+&-&-&0&+&0&0&+&+&+&+&0&0&0\\
+&-&+&-&0&0&+&0&-&+&-&0&0&+&0&-&-&+&0&0&0&+&+&-&-&0&+&0&0&+&+&+&+&0&0&0&0&0&0&-&-&-&-\\
+&-&-&+&0&0&0&+&-&+&-&0&0&+&0&+&-&-&0&+&0&0&+&+&+&+&0&0&0&0&0&0&-&-&-&-&-&-&+&0&0&0&+\\
+&0&0&0&-&-&-&-&-&+&-&0&0&+&0&+&+&+&+&0&0&0&0&0&0&-&-&-&-&-&-&+&0&0&0&+&+&-&-&0&+&0&0\\
+&+&+&+&+&0&0&0&-&+&-&0&0&+&0&-&+&-&0&0&+&0&-&+&-&0&0&+&0&-&+&-&0&0&+&0&-&+&-&0&0&+&0\\
+&+&-&-&0&+&0&0&-&-&+&0&0&0&+&+&+&+&+&0&0&0&+&-&-&0&+&0&0&0&0&0&-&-&-&-&-&+&-&0&0&+&0\\
+&-&+&-&0&0&+&0&-&-&+&0&0&0&+&+&-&-&0&+&0&0&0&0&0&-&-&-&-&-&+&-&0&0&+&0&+&+&+&+&0&0&0\\
+&-&-&+&0&0&0&+&-&-&+&0&0&0&+&0&0&0&-&-&-&-&-&+&-&0&0&+&0&+&+&+&+&0&0&0&+&-&-&0&+&0&0\\
+&0&0&0&-&-&-&-&-&-&+&0&0&0&+&-&+&-&0&0&+&0&+&+&+&+&0&0&0&+&-&-&0&+&0&0&0&0&0&-&-&-&-\\
+&+&+&+&+&0&0&0&-&-&+&0&0&0&+&-&-&+&0&0&0&+&-&-&+&0&0&0&+&-&-&+&0&0&0&+&-&-&+&0&0&0&+\\
+&+&-&-&0&+&0&0&0&0&0&-&-&-&-&-&-&+&0&0&0&+&+&+&+&+&0&0&0&-&+&-&0&0&+&0&+&-&-&0&+&0&0\\
+&-&+&-&0&0&+&0&0&0&0&-&-&-&-&+&+&+&+&0&0&0&-&+&-&0&0&+&0&+&-&-&0&+&0&0&-&-&+&0&0&0&+\\
+&-&-&+&0&0&0&+&0&0&0&-&-&-&-&-&+&-&0&0&+&0&+&-&-&0&+&0&0&-&-&+&0&0&0&+&+&+&+&+&0&0&0\\
+&0&0&0&-&-&-&-&0&0&0&-&-&-&-&+&-&-&0&+&0&0&-&-&+&0&0&0&+&+&+&+&+&0&0&0&-&+&-&0&0&+&0\\
+&+&+&+&+&0&0&0&0&0&0&-&-&-&-&0&0&0&-&-&-&-&0&0&0&-&-&-&-&0&0&0&-&-&-&-&0&0&0&-&-&-&-\\
+&+&-&-&0&+&0&0&+&+&+&+&0&0&0&+&-&-&0&+&0&0&-&+&-&0&0&+&0&-&-&+&0&0&0&+&0&0&0&-&-&-&-\\
+&-&+&-&0&0&+&0&+&+&+&+&0&0&0&-&+&-&0&0&+&0&-&-&+&0&0&0&+&0&0&0&-&-&-&-&+&-&-&0&+&0&0\\
+&-&-&+&0&0&0&+&+&+&+&+&0&0&0&-&-&+&0&0&0&+&0&0&0&-&-&-&-&+&-&-&0&+&0&0&-&+&-&0&0&+&0\\
+&0&0&0&-&-&-&-&+&+&+&+&0&0&0&0&0&0&-&-&-&-&+&-&-&0&+&0&0&-&+&-&0&0&+&0&-&-&+&0&0&0&+\\
+&+&+&+&+&0&0&0&+&+&+&+&0&0&0&+&+&+&+&0&0&0&+&+&+&+&0&0&0&+&+&+&+&0&0&0&+&+&+&+&0&0&0\\
  \end{array}\right).
  $\end{tiny}
 \end{defenum}
 Evidently, this resolves the question of existence of a $\w(43,25)$ given in Part V of \cite{handbook}.
\end{ex}

To further evince the utility of the method, we include a table of constructible parameters given a seed weighing matrix known to exist.

\begin{small}
\begin{longtable}[c]{rl|rl}
 \caption{Small parameter consequential order/weight pairs.\label{weighing-table}}\\

 %\hline
{\it Seed} $(v,k)$ & {\it Succident} $(v',k')$ & {\it Seed} $(v,k)$ & {\it Succident} $(v',k')$ \\
\hline
 \endfirsthead

 \multicolumn{4}{c}{\it Continuation of Table \ref{weighing-table}.}\\
 %\hline
 {\it Seed} $(v,k)$ & Succident $(v',k')$ & {\it Seed} $(v,k)$ & Succident $(v',k')$ \\
 \hline
 \endhead

 %\hline
 \endfoot

$(6,5)^\ddag$: & $(31,25)\ddag$, $(156,125)\ddag$, $(781,625)\ddag$ & $(16,3)$: & $(69,9)$, $(196,27)$, $(601,81)$ \\
$(8,5)$: & $(43,25)^\dag$, $(218,125)$ & $(16,5)$: & $(91,25)$, $(466,125)$ \\
$(8,7)^\ddag$: & $(57,49)^\ddag$, $(400,343)^\ddag$ & $(16,7)$: & $(121,49)$, $(856,343)$ \\
$(10,5)$: & $(55,25)$, $(280,125)$ & $(16,9)$: & $(151,81)$ \\
$(10,9)^\ddag$: & $(91,81)^\ddag$, $(820,729)^\ddag$ & $(16,11)$: & $(181,121)^\dag$ \\
$(12,5)$: & $(67,25)$, $(342,125)$ & $(16,13)$: & $(211,169)$ \\
$(12,7)$: & $(89,49)^\dag$, $(628,343)$ & $(18,13)$: & $(239,169)$ \\
$(12,9)$: & $(111,81)^\dag$ & $(19,9)^\ddag$: & $(181,81)^*$ \\
$(13,9)^\ddag$: & $(121,81)$ & $(20,7)$: & $(153,49)$ \\
$(14,9)$: & $(131,81)$ & $(20,13)$: & $(267,169)^\dag$ \\
$(14,13)^\ddag$: & $(183,169)^\ddag$ & & \\

\end{longtable}
\end{small}

$^*$ Note that the $\w(181,81)$ constructible from the balanced seed $\w(19,9)$ can be made to be balanced as shown in the previous chapter. It is not, however, a consequence of the construction of this chapter that the succident matrix is balanced.

$^\dag$ Denotes previously unknown order weight pairs.

$^\ddag$ Denotes a balanced weighing matrix.

\dinkus

% subsection %%%%%%%%%%%%%%%%%%%%%%%%%%%%%%%%%%%%%%%%%%%%%%%%%%%%%%%%%%%%%%%%%%%%%%%%%%%%%%%%
\subsection{A Second Application: Block designs}

The construction is perfectly amenable to certain symmetric designs. For precisely the same reasoning given in the previous subsection, we require that the design have a prime power block size. Furthermore, we require that the parameters of the residual of the design be invariant under complementation. Explicitly, if $A = \left(\begin{smallmatrix} \zz & R \\ \jj & D \end{smallmatrix}\right)$ is the incidence matrix of the given design, then $J-R$ must have the same parameters as $R$.

As shown in \cite{combinatorial-theory}, there is a $\w(2q+2,2q+2)$ if and only if there is a symmetric $\bibd(2q+1,q,(q-1)/2)$, called a {\it Hadamard design}\index{Hadamard design} \Fnote{hadamard-design}. Accordingly, the residual of this design is a $\bibd(q+1,2q,q,(q+1)/2,(q-1)/2)$, the parameters of which are invariant under complementation. Also, the derived design is a $\bibd(q,2q,q-1,(q-1)/2,(q-3)/2)$.

Let $H=H_0-H_1$ be a balanced $\w((q^n-1)/(q-1),q^{n-1})$, and let $\{A_\alpha\}_{\alpha\in\gf(q)}$ be as above. If $A = \left(\begin{smallmatrix} \zz & R \\ \jj & D \end{smallmatrix}\right)$ is a Hadamard design with parameters $(2q+1,q,(q-1)/2)$, then we form $\rr = H_0 \otimes R + H_1 \otimes (J-R)$. Indexing the rows of $D$ by the elements of $\gf(q)$, we then form $\DD = \sum_{\alpha\in\gf(q)} A_\alpha \otimes r_\alpha$.

We claim that $\A = \left(\begin{smallmatrix} \zz & \rr \\ \jj & \DD \end{smallmatrix}\right)$ is a symmetric design. To show this, we proceed as before. Using Lemma \ref{bgw-circ-lem}, we find
\begin{small}
\begin{align*}
 \rr\rr^t &= [H_0 \otimes R + H_1 \otimes (J_{q+1,2q}-R)][H_0 \otimes R + H_1 \otimes (J_{q+1,2q}-R)]^t \\
 &= (H_0H_0^t+H_1H_1^t) \otimes \left( \frac{q+1}{2}I_{q+1}+\frac{q-1}{2}J_{q+1} \right) \\&\qquad\qquad\qquad\qquad\qquad+ \frac{q+1}{2}(H_0H_1^t+H_1H_0^t) \otimes (J_{q+1}-I_{q+1}) \\
 &= \frac{q^{n-2}}{4}[(q+1)I_\frac{q^{n-1}}{q-1}+(q-1)J_\frac{q^{n-1}}{q-1}] \otimes [(q+1)I_{q+1}+(q-1)J_{q+1}] \\
 &\qquad\qquad\qquad\qquad\qquad+ \frac{q^{n-2}(q^2-1)}{4}(J_\frac{q^{n-1}}{q-1}-I_\frac{q^{n-1}}{q-1}) \otimes (J_{q+1}-I_{q+1}) \\
 &= \frac{q^{n-2}}{4}\left[ 2q(q+1)I_\frac{(q+1)(q^n-1)}{q-1} + 2q(q-1)J_\frac{(q+1)(q^n-1)}{q-1} \right] \\
 &= q^nI_\frac{(q+1)(q^n-1)}{q-1} + \frac{q^n-q^{n-1}}{2}(J_\frac{(q+1)(q^n-1)}{q-1}-I_\frac{(q+1)(q^n-1)}{q-1}),
\end{align*}
\end{small}
whence $\rr$ is a quasi-residual BIBD with parameters 
\[
\left(
\frac{(q+1)(q^n-1)}{q-1},\frac{2q^{n+1}-2q}{q-1},q^n,\frac{q^n+q^{n-1}}{2},\frac{q^n-q^{n-1}}{2}
\right).
\]
Next, by Lemma \ref{oa-lem}
\begin{align*}
 \DD\DD^t &= \left( \sum_\alpha A_\alpha \otimes r_\alpha \right)\left( \sum_\alpha A_\alpha \otimes r_\alpha \right)^t \\
 &= \sum_\alpha A_\alpha A_\alpha^t \otimes r_\alpha r_\alpha^t + \sum_{\alpha\neq\beta} A_\alpha A_\beta^t \otimes r_\alpha r_\beta^t \\
 &= (q-1)\sum_\alpha A_\alpha A_\alpha^t + \frac{q-3}{2}\sum_{\alpha\neq\beta} A_\alpha A_\beta^t \\
 &= (q-1)\left( \frac{q^{n-1}-1}{q-1}J - q^{n-1}I \right) + \frac{q^{n-1}(q-3)}{2}(J-I) \\
 &= (q^n-1)I + \frac{q^n-q^{n-1}-2}{2}(J-I),
\end{align*}
and $\DD$ is a quasi-derived BIBD with parameters
\[
 \left(
 q^{n+1}, \frac{2q^{n+1}-2q}{q-1}, q^n-1, \frac{q^n-q^{n-1}}{2}, \frac{q^n-q^{n-1}-2}{2}
 \right).
\]
Finally,
\begin{align*}
 \rr\DD^t &= [H_0 \otimes R + H_1 \otimes (J-R)] \left( \sum_\alpha A_\alpha \otimes r_\alpha \right)^t \\
 &= \sum_\alpha H_0A_\alpha^t \otimes Rr_\alpha^t + \sum_\alpha H_1A_\alpha^t \otimes (J-R)r_\alpha^t \\
 &= \frac{q-1}{2}\sum_\alpha H_0A_\alpha^t \otimes \jj + \frac{q-1}{2}\sum_\alpha H_1A_\alpha^t \otimes \jj \\
 &= \frac{q-1}{2}(H_0+H_1)\sum_\alpha A_\alpha^t \otimes \jj \\
 &= \frac{q-1}{2}(H_0+H_1)J \\
 &= \frac{q^n-q^{n-1}}{2}J.
\end{align*}
We have shown that $\A = \left(\begin{smallmatrix} \zz & \rr \\ \jj & \DD \end{smallmatrix}\right)$ is a symmetric BIBD. We record this result below.

\begin{thm}
 If there is a symmetric $\bibd(2q+1,q,(q-1)/2)$, then there is a symmetric BIBD with parameters
 \begin{equation}
  \left(
  \frac{2q^{n+1}-2q}{q-1}+1, q^n, \frac{q^n-q^{n-1}}{2}
  \right).
 \end{equation}
\end{thm}

\begin{ex}
 Consider the Hadamard design
 \begin{defenum}
  \item $
  \arraycolsep=2.0pt\def\arraystretch{0.5}
  \left(\begin{array}{ccccccc}
0&1&1&0&1&0&0\\
0&0&1&1&0&1&0\\
0&0&0&1&1&0&1\\
1&0&0&0&1&1&0\\
0&1&0&0&0&1&1\\
1&0&1&0&0&0&1\\
1&1&0&1&0&0&0\\
  \end{array}\right)
  $
 \end{defenum}
 with parameters $(7,3,1)$. The result of the first iteration of the construction is the symmetric $\bibd(25,9,3)$
 \begin{defenum}[resume]
  \item $
  \arraycolsep=2.0pt\def\arraystretch{0.5}
  \left(\begin{array}{ccccccccccccccccccccccccc}
0&0&0&0&0&0&0&1&1&0&1&0&0&1&1&0&1&0&0&1&1&0&1&0&0\\
0&0&0&0&0&0&0&0&1&1&0&1&0&0&1&1&0&1&0&0&1&1&0&1&0\\
0&0&0&0&0&0&0&0&0&1&1&0&1&0&0&1&1&0&1&0&0&1&1&0&1\\
0&0&0&0&0&0&0&1&0&0&0&1&1&1&0&0&0&1&1&1&0&0&0&1&1\\
0&0&0&1&0&1&1&0&0&0&0&0&0&0&0&1&0&1&1&1&1&0&1&0&0\\
0&1&0&0&1&0&1&0&0&0&0&0&0&1&0&0&1&0&1&0&1&1&0&1&0\\
0&1&1&0&0&1&0&0&0&0&0&0&0&1&1&0&0&1&0&0&0&1&1&0&1\\
0&0&1&1&1&0&0&0&0&0&0&0&0&0&1&1&1&0&0&1&0&0&0&1&1\\
0&0&0&1&0&1&1&1&1&0&1&0&0&0&0&0&0&0&0&0&0&1&0&1&1\\
0&1&0&0&1&0&1&0&1&1&0&1&0&0&0&0&0&0&0&1&0&0&1&0&1\\
0&1&1&0&0&1&0&0&0&1&1&0&1&0&0&0&0&0&0&1&1&0&0&1&0\\
0&0&1&1&1&0&0&1&0&0&0&1&1&0&0&0&0&0&0&0&1&1&1&0&0\\
0&0&0&1&0&1&1&0&0&1&0&1&1&1&1&0&1&0&0&0&0&0&0&0&0\\
0&1&0&0&1&0&1&1&0&0&1&0&1&0&1&1&0&1&0&0&0&0&0&0&0\\
0&1&1&0&0&1&0&1&1&0&0&1&0&0&0&1&1&0&1&0&0&0&0&0&0\\
0&0&1&1&1&0&0&0&1&1&1&0&0&1&0&0&0&1&1&0&0&0&0&0&0\\
1&0&1&0&0&0&1&0&1&0&0&0&1&1&0&1&0&0&0&0&0&0&1&1&0\\
1&1&0&1&0&0&0&0&1&0&0&0&1&0&0&0&1&1&0&1&0&1&0&0&0\\
1&0&0&0&1&1&0&0&1&0&0&0&1&0&1&0&0&0&1&0&1&0&0&0&1\\
1&0&1&0&0&0&1&1&0&1&0&0&0&0&0&0&1&1&0&0&1&0&0&0&1\\
1&1&0&1&0&0&0&1&0&1&0&0&0&0&1&0&0&0&1&0&0&0&1&1&0\\
1&0&0&0&1&1&0&1&0&1&0&0&0&1&0&1&0&0&0&1&0&1&0&0&0\\
1&0&1&0&0&0&1&0&0&0&1&1&0&0&1&0&0&0&1&1&0&1&0&0&0\\
1&1&0&1&0&0&0&0&0&0&1&1&0&1&0&1&0&0&0&0&1&0&0&0&1\\
1&0&0&0&1&1&0&0&0&0&1&1&0&0&0&0&1&1&0&0&0&0&1&1&0\\
  \end{array}\right).
  $
 \end{defenum}
\end{ex}

\biblio
\end{document}
