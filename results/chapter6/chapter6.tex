%\providecommand{\main}{../../main}
\documentclass[../../main]{subfiles}

\newendnotes{f}
\input{ch6notes.txt}
\setcounter{fnote}{\value{enote}}
% \newcommand{\Fnote}[1]{\fnote{#1}\big)}
\newcommand{\Fnote}[1]{\fnote{#1}}

\begin{document}

In this final chapter of results, we present a general and novel method
of constructing certain configurations using the classical parameter BGW
matrices and the generalized simplex codes \cite[see][]{w-mat-construct}. The
procedure is new and does not appear to be previously contained in the
literature of the field. Parts of the method are similar to that given by
\cite{rajkundlia} and \cite{ionin-bgw-bibd}, but it can be seen that it is much
simpler to apply than the recursive methods of the aforementioned seminal
articles. Following the presentation of this construction technique, we find an
equivalence between arbitrary BGW matrices over a finite abelian group and
certain commutative association schemes. This result generalizes those obtained
over the course of the previous chapter. The results presented here are a
modified version of \cite{w-mat-construct}. 

\fancyhf{}

\fancyhead[RO,LE]{\thepage}
% \fancyfoot[C]{\thepage}
\fancyhead[CO]{\S\thesection. A General Method}
\fancyhead[CE]{Chapter \thechapter. A Unified Construction of Weighing Matrices}

 \section{\centering A General Method}
 
 In this section, the general method intimated above is derived and used in the
 construction of weighing matrices and symmetric designs. Resulting parameters
 of configurations so constructible are then tabulated. 
 
 \dinkus
 
 \subfile{./genmethod/genmethod}
 
 \fancyhf{}

 \fancyhead[RO,LE]{\thepage}
 % \fancyfoot[C]{\thepage}
 \fancyhead[CO]{\S\thesection. BGW Matrices and Association Schemes}
 \fancyhead[CE]{Chapter \thechapter. A Unified Construction of Weighing Matrices}

 \section{\centering BGW Matrices and Association Schemes}
 
 In this section we present perhaps the most striking result of this thesis. Here
 the equivalence between BGW matrices and commutative association schemes is
 described in detail. 
 
 \dinkus
 
 \subfile{./appschemes/appschemes}
 
 \singlespace
 
 \fancyhf{}

 \fancyhead[RO,LE]{\thepage}
 % \fancyfoot[C]{\thepage}
 \fancyhead[CO]{Notes}
 \fancyhead[CE]{Chapter \thechapter. A Unified Construction of Weighing Matrices}

 \addcontentsline{toc}{section}{Notes}
 \section*{\centering Notes}
 \thefnotes
 
 \doublespacing
 
 \biblio
\end{document}
