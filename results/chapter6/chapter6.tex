%\providecommand{\main}{../../main}
\documentclass[../../main]{subfiles}

\newendnotes{f}
\input{ch6notes.txt}
\setcounter{fnote}{\value{enote}}
\newcommand{\Fnote}[1]{\fnote{#1}\big)}

\begin{document}

 In this final chapter of results, we present a general method to construct certain configurations using classical parameter BGW matrices and the generalized simplex codes. This method is similar to that given by \cite{rajkundlia} and \cite{ionin-bgw-bibd}, but it can be seen that it is much simpler to apply than the recursive methods of the aforementioned seminal articles. Following the presentation of this construction technique, we find an equivalence between arbitrary BGW matrices over a finite abelian group and certain commutative association schemes. This result generalizes that obtained over the course of the previous chapter.

 \section{\centering A General Method}
 
 In this section, the general method intimated above is derived and used in the construction of weighing matrices and symmetric designs. Resulting parameters of configurations so constructible are then tabulated.
 
 \dinkus
 
 \subfile{./genmethod/genmethod}
 
 \section{\centering BGW Matrices and Association Schemes}
 
 In this section we present perhaps the most striking result of this essay. Here the equivalence between BGW matrices and commutative association schemes is described in detail.
 
 \dinkus
 
 \subfile{./appschemes/appschemes}
 
 \singlespace
 
 \addcontentsline{toc}{section}{Notes}
 \section*{\centering Notes}
 \thefnotes
 
 \doublespacing
 
 \biblio
\end{document}
