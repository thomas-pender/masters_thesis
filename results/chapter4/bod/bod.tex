\documentclass[../../../main]{subfiles}

\begin{document}

% subsection %%%%%%%%%%%%%%%%%%%%%%%%%%%%%%%%%%%%%%%%%%%%%%%%%%%%%%%%%%%%%%%%%%%%%%%%%%%%%%%%
\subsection{Definition}

Generalizing balanced splittability to orthogonal designs presents several difficulties. The various cases are encapsulated in the next definition.

\begin{defin}\index{balancedly splittable orthogonal designs}
 Let $X$ be a full $\qod(n;s_1, \dots, s_u)$. $X$ is {\it balancedly splittable} if there is an $\ell \times n$ submatrix $X_1$ where one of the following conditions holds. In what follows $\alpha, \beta \in \{a+ib+jc+kd : a,b,c,d \in \R\}$.
 \begin{defenum}
  \item The off-diagonal entries of $X_1^*X_1$ are in the set
  \begin{small}
  \[
   \{\pm\varepsilon cx_1^{m_1} \cdots x_u^{m_u}x_1^{*m_1'} \cdots x_u^{*m_u'} : m_i,m_i' \in \N, \varepsilon \in \{1,i,j,k\}, c \in \{\alpha,\alpha^*,\beta,\beta^*\}\}.
  \]
  \end{small}
  
  \item\label{od-split-2} The off-diagonal entries of $X_1^*X_1$ are in the set
  \[
   \left\{
   \sum_{i=1}^u t_i|x_i|^2 : t_i \in \N, \sum_{i=0}^u t_i=m
   \right\}
  \]
  or in the set
  \[
   \{\pm\varepsilon cx_1^{m_1} \cdots x_u^{m_u}x_1^{*m_1'} \cdots x_u^{*m_u'} : m_i,m_i' \in \N, \varepsilon \in \{1,i,j,k\}, c \in \{\alpha,\alpha^*,\beta,\beta^*\}\}.
  \]
  
  \item The off-diagonal entries of $X_1^*X_1$ are in the set 
  \[
   \{\pm\varepsilon c\sigma : \varepsilon \in \{1,i,j,k\}, c \in \{\alpha,\alpha^*,\beta,\beta^*\}\},
  \]
  where $\sigma = \ssum_i s_i|x_i|^2$ (cf. \ref{qod-od-def}).
 \end{defenum}
 In the first case, the split is {\it unstable}; in the second, the split is {\it unfaithfully unstable}; and in the third, the split is {\it stable}. The term {\it faithful} is used to describe the first and third cases.
\end{defin}

From the definition, we see that if $\alpha$ and $\beta$ are the same in absolute value, the split corresponds to a set of equiangular lines. Interestingly, we will see that both conditions in \ref{od-split-2} can hold simultaneously.

The next two subsections will present constructions for both the unfaithful, and the faithful case.

\dinkus

% subsection %%%%%%%%%%%%%%%%%%%%%%%%%%%%%%%%%%%%%%%%%%%%%%%%%%%%%%%%%%%%%%%%%%%%%%%%%%%%%%%%
\subsection{Unfaithful Constructions}

The constructions of this section are similar to those presented in \cite{fender-quh} and \cite{pender_2020}, and are applicable to real and complex orthogonal designs.

To begin, if $W$ is a skew-symmetric $\w(q+1,q)$, then we take $Q$ to be its core, i.e. the submatrix obtaind by deleting the first row and column. Further, we can assume that $W=\left(\begin{smallmatrix} 0 & \jj^t \\ -\jj & Q \end{smallmatrix}\right)$, hence $JQ=QJ=O$ and $Q^2=J-qI$. We recursively define the following family of matrices.
\begin{align*}
 \J_m&=\begin{cases}
        aJ_1 & \text{if $m=0$, and} \\
        J_q \otimes \A_{m-1} & \text{if $m>0$;}
       \end{cases} \\
 \A_m&=\begin{cases}
      bJ_1 & \text{if $m=0$, and} \\
      I_q \otimes \J_{m-1} + Q \otimes \A_{m-1} & \text{ if $m>0$;}
     \end{cases}
\end{align*}
where $a$ and $b$ are commuting indeterminants.

We require the following lemma.

\begin{lem}
 \begin{defenum}
  \item[]
  \item\label{lem-am-1} $\J_m\A_m^t = \A_m\J_m^t$;
  \item\label{lem-am-2} $\J_m\J_m^t + q\A_m\A_m^t = (q^ma^2+q^{m+1}b^2)I$; and
  \item\label{lem-am-3} $\J_1^t\J_1 = qa^2J, \A_1^t\A_1=a^2I+b^2(qI-J)$, and $\A_1^t\J_1=\J_1^t\A_1=abJ$.
 \end{defenum}
\end{lem}

\begin{proof}
 We have $\J_0\A_0^t=ab=ba=\A_0\J_0^t$. Assume $\J_{m-1}\A_{m-1}^t=\A_{m-1}\J_{m-1}^t$. Then
 \begin{align*}
  \J_m\A_m^t &= (J \otimes \A_{m-1})(I \otimes \J_{m-1} + Q \otimes \A_{m-1})^t \\
  &= J \otimes \A_{m-1}\J_{m-1}^t + JQ^t \otimes \A_{m-1}\A_{m-1}^t \\
  &= J \otimes \J_{m-1}\A_{m-1}^t + QJ \otimes \A_{m-1}\A_{m-1}^t \\
  &= (I \otimes \J_{m-1} + Q \otimes \A_{m-1})(J \otimes \A_{m-1})^t,
 \end{align*}
 and \ref{lem-am-1} has been shown.
 
 Clearly, $\J_0\J_0^t + q\A_0\A_0^t = a^2+qb^2$; so, assume $\J_{m-1}\J_{m-1}^t + q\A_{m-1}\A_{m-1}^t = (a^2+qb^2)I$. Then
 \begin{small}
 \begin{align*}
  \J_m\J_m^t + q\A_m\A_m^t &= qJ \otimes \A_{m-1}\A_{m-1}^t + q(I \otimes \J_{m-1}\J_{m-1}^t - Q^2 \otimes \A_{m-1}\A_{m-1}^t) \\
  &= qI \otimes (\J_{m-1}\J_{m-1}^t + q\A_{m-1}\A_{m-1}^t) \\
  &= qI \otimes (q^{m-1}q^2 + q^mb^2)I \\
  &= (q^ma^2 + q^{m+1}b^2)I,
 \end{align*}
 \end{small}
 and \ref{lem-am-2} is proven.
 
 Finally, \ref{lem-am-3} is simply a restatement of the definitions of $\J_m$ and $\A_m$.
\end{proof}

We can now present the first construction of the novel balancedly splittable ODs.

\begin{thm}
 Let $W, \A_m$, and $\J_m$ be as above. Define $X_m = I \otimes \J_m + W \otimes \A_m$. Then:
 \begin{defenum}
  \item $X_m$ is an $\od(q^m(q+1); q^m, q^{m+1})$, and
  \item The matrix $X_1$ is an unfaithful balancedly splittable $\od(q(q+1); q, q^2)$.
 \end{defenum}
\end{thm}

\begin{proof}
 $X_m$ has entries from $\{\pm a, \pm b\}$. Observe:
 \begin{align*}
  X_mX_m^t &= (I \otimes \J_m + W \otimes \A_m)(I \otimes \J_m + W \otimes \A_m)^t \\
  &= I \otimes \J_m\J_m^t + WW^t \otimes \A_m\A_m^t \\
  &= I \otimes \J_m\J_m^t + qI \otimes \A_m\A_m^t \\
  &= I \otimes (\J_m\J_m^t + q\A_m\A_m^t) \\
  &= I \otimes (q^ma^2 + q^{m+1}b^2)I \\
  &= (q^ma^2 + q^{m+1}b^2)I,
 \end{align*}
 which shows that $X_m$ is an $\od(q^m(q+1); q^m; q^{m+1})$. It remains to prove the balanced splittability of the base case.
 
 Take $Y=(\begin{smallmatrix} \J_1 & \A_1 & \dots & \A_1 \end{smallmatrix})$, the first block row of $X_1$. Then
 \begin{align*}
  Y^tY &= \begin{pmatrix} \J_1^t\J_1 & \jj^t \otimes \J_1^t\A_1 \\ \jj \otimes \A_1^t\J_1 & J \otimes \A_1^t\A_1 \end{pmatrix} \\
  &= \begin{pmatrix} qa^2J & ab\jj^t \otimes J \\ ab\jj \otimes J & J \otimes [(a^2-b^2)J + qb^2I] \end{pmatrix}.
 \end{align*}
 Hence, $X_1$ admits an unfaithfully balanced split.
\end{proof}

\begin{cor}
 For every prime power $q \equiv -1 \pmod{4}$, and for every integer $m>0$, there is an $\od(q^m(q+1); q^m, q^{m+1})$
\end{cor}

\begin{proof}
 By Propositions \ref{gc proposition} and \ref{gbrd epimorphism}, there is a skew-symmetric $\w(q+1,q)$. Apply the theorem to this matrix.
\end{proof}

\begin{cor}
 For every prime power $q \equiv -1 \pmod{4}$, there is an unfaithful balancedly splittable $\od(q(q+1); q, q^2)$. 
\end{cor}

\begin{ex}
 Using the skew-symmetric Paley weighing matrix\Dnote{paley-note} $W(4,3)$ given by
 \begin{defenum}
  \item $
  \arraycolsep=1.25pt\def\arraystretch{0.625}
  \left(\begin{array}{cccc}
  0&+&+&+\\
  -&0&+&-\\
  -&-&0&+\\
  -&+&-&0
  \end{array}\right),
  $
 \end{defenum}
 we construct the smallest case of an $\od(12; 3, 9)$ given by the theorem
 \begin{defenum}[resume]
  \item $
  %\arraycolsep=1.25pt\def\arraystretch{0.625}
  \arraycolsep=2.0pt\def\arraystretch{0.5}
  \left(\begin{array}{cccccccccccc}
   \mathbf{b} & \mathbf{b} & \mathbf{b} & \mathbf{a} & \mathbf{b} & \mathbf{\bar b} & \mathbf{a} & \mathbf{b} & \mathbf{\bar b} & \mathbf{a} & \mathbf{b} & \mathbf{\bar b} \\
\mathbf{b} & \mathbf{b} & \mathbf{b} & \mathbf{\bar b} & \mathbf{a} & \mathbf{b} & \mathbf{\bar b} & \mathbf{a} & \mathbf{b} & \mathbf{\bar b} & \mathbf{a} & \mathbf{b} \\
\mathbf{b} & \mathbf{b} & \mathbf{b} & \mathbf{b} & \mathbf{\bar b} & \mathbf{a} & \mathbf{b} & \mathbf{\bar b} & \mathbf{a} & \mathbf{b} & \mathbf{\bar b} & \mathbf{a} \\
\mathbf{\bar a} & \mathbf{\bar b} & \mathbf{b} & \mathbf{b} & \mathbf{b} & \mathbf{b} & \mathbf{a} & \mathbf{b} & \mathbf{\bar b} & \mathbf{\bar a} & \mathbf{\bar b} & \mathbf{b} \\
b & \bar a & \bar b & b & b & b & \bar b & a & b & b & \bar a & \bar b \\
\bar b & b & \bar a & b & b & b & b & \bar b & a & \bar b & b & \bar a \\
\bar a & \bar b & b & \bar a & \bar b & b & b & b & b & a & b & \bar b \\
b & \bar a & \bar b & b & \bar a & \bar b & b & b & b & \bar b & a & b \\
\bar b & b & \bar a & \bar b & b & \bar a & b & b & b & b & \bar b & a \\
\bar a & \bar b & b & a & b & \bar b & \bar a & \bar b & b & b & b & b \\
b & \bar a & \bar b & \bar b & a & b & b & \bar a & \bar b & b & b & b \\
\bar b & b & \bar a & b & \bar b & a & \bar b & b & \bar a & b & b & b \\
  \end{array}\right),
  $
 \end{defenum}
 where the unfaithful split is shown in bold.
\end{ex}

Our first construction yields real ODs and is applicable in the case that we have a prime power $q \equiv -1 \pmod{4}$. Of course, since $(q-1)/2$ is odd, we can apply the results of $\S4$ to construct a weighing matrix that is skew-symmetric, a property essential to the construction. If $q \equiv 1 \pmod{4}$, then $(q-1)/2$ is even and the ensuing weighing matrix is symmetric. In this event, we need to appeal to complex ODs in order apply the construction.

To apply the complex units, we make the following recursive definitions where $W=\left(\begin{smallmatrix} 0 & \jj^t \\ \jj & Q \end{smallmatrix}\right)$ is a $\w(q+1,q)$ with $Q^t=Q$.
\begin{align*}
 \CC_m&=\begin{cases}
        aJ_1 & \text{if $m=0$, and} \\
        J_q \otimes \DD_{m-1} & \text{if $m>0$;}
       \end{cases}\\
 \DD_m&=\begin{cases}
        bJ_1 & \text{if $m=0$, and} \\
        I_q \otimes \CC_{m-1} + iQ \otimes \DD_{m-1} & \text{if $m>0$;}
       \end{cases}
\end{align*}
where again $a$ and $b$ are real commuting indeterminants. As above, we have the following lemma that is shown in precisely the same way as before, save one replaces transposition with conjugate transposition.

\begin{lem}
  \begin{defenum}
  \item[]
  \item $\CC_m\DD_m^* = \DD_m\CC_m^*$;
  \item $\CC_m\CC_m^* + q\DD_m\DD_m^* = (q^ma^2+q^{m+1}b^2)I$; and
  \item $\CC_1^*\CC_1 = qa^2J, \DD_1^*\DD_1=a^2I+b^2(qI-J)$, and $\DD_1^*\CC_1=\CC_1^*\DD_1=abJ$.
 \end{defenum}
\end{lem}

\begin{thm}
 Let $W$, $\CC_m$, and $\DD_m$ be as above, and define $Y_m=iI \otimes \CC_m + W \otimes \DD_m$. Then:
 \begin{defenum}
  \item The matrix $Y_m$ is a $\cod(q^m(q+1); q^m, q^{m+1})$, and
  \item $Y_1$ admits an unfaithfully balanced split.
 \end{defenum}
\end{thm}

\begin{cor}
 For every prime power $q \equiv 1 \pmod{4}$, and for every integer $m>0$, there is a $\cod(q^m(q+1); q^m, q^{m+1})$.
\end{cor}

\begin{cor}
 For every prime power $q \equiv 1 \pmod{4}$, there is an unfaithful balancedly splittable $\cod(q(q+1); q, q^2)$.
\end{cor}

We again explore the smallest case.

\begin{ex}
 Take $q=5$ and consider the symmetric Paley weighing matrix $\w(6,5)$ given by
 \begin{defenum}
  \item $
  \arraycolsep=1.25pt\def\arraystretch{0.625}
  \left(\begin{array}{cccccc}
   0&+&+&+&+&+\\
   +&0&+&-&-&+\\
   +&+&0&+&-&-\\
   +&-&+&0&+&-\\
   +&-&-&+&0&+\\
   +&+&-&-&+&0
  \end{array}\right).
  $
 \end{defenum}
 Applying the construction, we obtain
 \begin{defenum}[resume]
  \item \begin{tiny}$
  \arraycolsep=1.25pt\def\arraystretch{0.625}
  \left(\begin{array}{cccccccccccccccccccccccccccccc}
   \mathbf{ib}&\mathbf{ib}&\mathbf{ib}&\mathbf{ib}&\mathbf{ib}&\mathbf{a}&\mathbf{ib}&\mathbf{i\bar b}&\mathbf{i\bar b}&\mathbf{ib}&\mathbf{a}&\mathbf{ib}&\mathbf{i\bar b}&\mathbf{i\bar b}&\mathbf{ib}&\mathbf{a}&\mathbf{ib}&\mathbf{i\bar b}&\mathbf{i\bar b}&\mathbf{ib}&\mathbf{a}&\mathbf{ib}&\mathbf{i\bar b}&\mathbf{i\bar b}&\mathbf{ib}&\mathbf{a}&\mathbf{ib}&\mathbf{i\bar b}&\mathbf{i\bar b}&\mathbf{ib}\\
\mathbf{ib}&\mathbf{ib}&\mathbf{ib}&\mathbf{ib}&\mathbf{ib}&\mathbf{ib}&\mathbf{a}&\mathbf{ib}&\mathbf{i\bar b}&\mathbf{i\bar b}&\mathbf{ib}&\mathbf{a}&\mathbf{ib}&\mathbf{i\bar b}&\mathbf{i\bar b}&\mathbf{ib}&\mathbf{a}&\mathbf{ib}&\mathbf{i\bar b}&\mathbf{i\bar b}&\mathbf{ib}&\mathbf{a}&\mathbf{ib}&\mathbf{i\bar b}&\mathbf{i\bar b}&\mathbf{ib}&\mathbf{a}&\mathbf{ib}&\mathbf{i\bar b}&\mathbf{i\bar b}\\
\mathbf{ib}&\mathbf{ib}&\mathbf{ib}&\mathbf{ib}&\mathbf{ib}&\mathbf{i\bar b}&\mathbf{ib}&\mathbf{a}&\mathbf{ib}&\mathbf{i\bar b}&\mathbf{i\bar b}&\mathbf{ib}&\mathbf{a}&\mathbf{ib}&\mathbf{i\bar b}&\mathbf{i\bar b}&\mathbf{ib}&\mathbf{a}&\mathbf{ib}&\mathbf{i\bar b}&\mathbf{i\bar b}&\mathbf{ib}&\mathbf{a}&\mathbf{ib}&\mathbf{i\bar b}&\mathbf{i\bar b}&\mathbf{ib}&\mathbf{a}&\mathbf{ib}&\mathbf{i\bar b}\\
\mathbf{ib}&\mathbf{ib}&\mathbf{ib}&\mathbf{ib}&\mathbf{ib}&\mathbf{i\bar b}&\mathbf{i\bar b}&\mathbf{ib}&\mathbf{a}&\mathbf{ib}&\mathbf{i\bar b}&\mathbf{i\bar b}&\mathbf{ib}&\mathbf{a}&\mathbf{ib}&\mathbf{i\bar b}&\mathbf{i\bar b}&\mathbf{ib}&\mathbf{a}&\mathbf{ib}&\mathbf{i\bar b}&\mathbf{i\bar b}&\mathbf{ib}&\mathbf{a}&\mathbf{ib}&\mathbf{i\bar b}&\mathbf{i\bar b}&\mathbf{ib}&\mathbf{a}&\mathbf{ib}\\
\mathbf{ib}&\mathbf{ib}&\mathbf{ib}&\mathbf{ib}&\mathbf{ib}&\mathbf{ib}&\mathbf{i\bar b}&\mathbf{i\bar b}&\mathbf{ib}&\mathbf{a}&\mathbf{ib}&\mathbf{i\bar b}&\mathbf{i\bar b}&\mathbf{ib}&\mathbf{a}&\mathbf{ib}&\mathbf{i\bar b}&\mathbf{i\bar b}&\mathbf{ib}&\mathbf{a}&\mathbf{ib}&\mathbf{i\bar b}&\mathbf{i\bar b}&\mathbf{ib}&\mathbf{a}&\mathbf{ib}&\mathbf{i\bar b}&\mathbf{i\bar b}&\mathbf{ib}&\mathbf{a}\\
a&ib&i\bar b&i\bar b&ib&ib&ib&ib&ib&ib&a&ib&i\bar b&i\bar b&ib&\bar a&i\bar b&ib&ib&i\bar b&\bar a&i\bar b&ib&ib&i\bar b&a&ib&i\bar b&i\bar b&ib\\
ib&a&ib&i\bar b&i\bar b&ib&ib&ib&ib&ib&ib&a&ib&i\bar b&i\bar b&i\bar b&\bar a&i\bar b&ib&ib&i\bar b&\bar a&i\bar b&ib&ib&ib&a&ib&i\bar b&i\bar b\\
i\bar b&ib&a&ib&i\bar b&ib&ib&ib&ib&ib&i\bar b&ib&a&ib&i\bar b&ib&i\bar b&\bar a&i\bar b&ib&ib&i\bar b&\bar a&i\bar b&ib&i\bar b&ib&a&ib&i\bar b\\
i\bar b&i\bar b&ib&a&ib&ib&ib&ib&ib&ib&i\bar b&i\bar b&ib&a&ib&ib&ib&i\bar b&\bar a&i\bar b&ib&ib&i\bar b&\bar a&i\bar b&i\bar b&i\bar b&ib&a&ib\\
ib&i\bar b&i\bar b&ib&a&ib&ib&ib&ib&ib&ib&i\bar b&i\bar b&ib&a&i\bar b&ib&ib&i\bar b&\bar a&i\bar b&ib&ib&i\bar b&\bar a&ib&i\bar b&i\bar b&ib&a\\
a&ib&i\bar b&i\bar b&ib&a&ib&i\bar b&i\bar b&ib&ib&ib&ib&ib&ib&a&ib&i\bar b&i\bar b&ib&\bar a&i\bar b&ib&ib&i\bar b&\bar a&i\bar b&ib&ib&i\bar b\\
ib&a&ib&i\bar b&i\bar b&ib&a&ib&i\bar b&i\bar b&ib&ib&ib&ib&ib&ib&a&ib&i\bar b&i\bar b&i\bar b&\bar a&i\bar b&ib&ib&i\bar b&\bar a&i\bar b&ib&ib\\
i\bar b&ib&a&ib&i\bar b&i\bar b&ib&a&ib&i\bar b&ib&ib&ib&ib&ib&i\bar b&ib&a&ib&i\bar b&ib&i\bar b&\bar a&i\bar b&ib&ib&i\bar b&\bar a&i\bar b&ib\\
i\bar b&i\bar b&ib&a&ib&i\bar b&i\bar b&ib&a&ib&ib&ib&ib&ib&ib&i\bar b&i\bar b&ib&a&ib&ib&ib&i\bar b&\bar a&i\bar b&ib&ib&i\bar b&\bar a&i\bar b\\
ib&i\bar b&i\bar b&ib&a&ib&i\bar b&i\bar b&ib&a&ib&ib&ib&ib&ib&ib&i\bar b&i\bar b&ib&a&i\bar b&ib&ib&i\bar b&\bar a&i\bar b&ib&ib&i\bar b&\bar a\\
a&ib&i\bar b&i\bar b&ib&\bar a&i\bar b&ib&ib&i\bar b&a&ib&i\bar b&i\bar b&ib&ib&ib&ib&ib&ib&a&ib&i\bar b&i\bar b&ib&\bar a&i\bar b&ib&ib&i\bar b\\
ib&a&ib&i\bar b&i\bar b&i\bar b&\bar a&i\bar b&ib&ib&ib&a&ib&i\bar b&i\bar b&ib&ib&ib&ib&ib&ib&a&ib&i\bar b&i\bar b&i\bar b&\bar a&i\bar b&ib&ib\\
i\bar b&ib&a&ib&i\bar b&ib&i\bar b&\bar a&i\bar b&ib&i\bar b&ib&a&ib&i\bar b&ib&ib&ib&ib&ib&i\bar b&ib&a&ib&i\bar b&ib&i\bar b&\bar a&i\bar b&ib\\
i\bar b&i\bar b&ib&a&ib&ib&ib&i\bar b&\bar a&i\bar b&i\bar b&i\bar b&ib&a&ib&ib&ib&ib&ib&ib&i\bar b&i\bar b&ib&a&ib&ib&ib&i\bar b&\bar a&i\bar b\\
ib&i\bar b&i\bar b&ib&a&i\bar b&ib&ib&i\bar b&\bar a&ib&i\bar b&i\bar b&ib&a&ib&ib&ib&ib&ib&ib&i\bar b&i\bar b&ib&a&i\bar b&ib&ib&i\bar b&\bar a\\
a&ib&i\bar b&i\bar b&ib&\bar a&i\bar b&ib&ib&i\bar b&\bar a&i\bar b&ib&ib&i\bar b&a&ib&i\bar b&i\bar b&ib&ib&ib&ib&ib&ib&a&ib&i\bar b&i\bar b&ib\\
ib&a&ib&i\bar b&i\bar b&i\bar b&\bar a&i\bar b&ib&ib&i\bar b&\bar a&i\bar b&ib&ib&ib&a&ib&i\bar b&i\bar b&ib&ib&ib&ib&ib&ib&a&ib&i\bar b&i\bar b\\
i\bar b&ib&a&ib&i\bar b&ib&i\bar b&\bar a&i\bar b&ib&ib&i\bar b&\bar a&i\bar b&ib&i\bar b&ib&a&ib&i\bar b&ib&ib&ib&ib&ib&i\bar b&ib&a&ib&i\bar b\\
i\bar b&i\bar b&ib&a&ib&ib&ib&i\bar b&\bar a&i\bar b&ib&ib&i\bar b&\bar a&i\bar b&i\bar b&i\bar b&ib&a&ib&ib&ib&ib&ib&ib&i\bar b&i\bar b&ib&a&ib\\
ib&i\bar b&i\bar b&ib&a&i\bar b&ib&ib&i\bar b&\bar a&i\bar b&ib&ib&i\bar b&\bar a&ib&i\bar b&i\bar b&ib&a&ib&ib&ib&ib&ib&ib&i\bar b&i\bar b&ib&a\\
a&ib&i\bar b&i\bar b&ib&a&ib&i\bar b&i\bar b&ib&\bar a&i\bar b&ib&ib&i\bar b&\bar a&i\bar b&ib&ib&i\bar b&a&ib&i\bar b&i\bar b&ib&ib&ib&ib&ib&ib\\
ib&a&ib&i\bar b&i\bar b&ib&a&ib&i\bar b&i\bar b&i\bar b&\bar a&i\bar b&ib&ib&i\bar b&\bar a&i\bar b&ib&ib&ib&a&ib&i\bar b&i\bar b&ib&ib&ib&ib&ib\\
i\bar b&ib&a&ib&i\bar b&i\bar b&ib&a&ib&i\bar b&ib&i\bar b&\bar a&i\bar b&ib&ib&i\bar b&\bar a&i\bar b&ib&i\bar b&ib&a&ib&i\bar b&ib&ib&ib&ib&ib\\
i\bar b&i\bar b&ib&a&ib&i\bar b&i\bar b&ib&a&ib&ib&ib&i\bar b&\bar a&i\bar b&ib&ib&i\bar b&\bar a&i\bar b&i\bar b&i\bar b&ib&a&ib&ib&ib&ib&ib&ib\\
ib&i\bar b&i\bar b&ib&a&ib&i\bar b&i\bar b&ib&a&i\bar b&ib&ib&i\bar b&\bar a&i\bar b&ib&ib&i\bar b&\bar a&ib&i\bar b&i\bar b&ib&a&ib&ib&ib&ib&ib\\
  \end{array}\right),
  $\end{tiny}
 \end{defenum}
 a $\cod(30;5,25)$, where the unfaithful split is shown in bold.
\end{ex}

\dinkus

% subsection %%%%%%%%%%%%%%%%%%%%%%%%%%%%%%%%%%%%%%%%%%%%%%%%%%%%%%%%%%%%%%%%%%%%%%%%%%%%%%%%
\subsection{Faithful Construction}

Here we will introduce a most useful construction of orthogonal designs admitting a faithful split. We will see later how we can use these constructed matrices in constructing other objects.

To begin, we assume the existence of a full $\od(n; s_1, \dots, s_u)$, say $X$, and label the rows of $X$ as $x_0, \dots, x_{n-1}$. Further, assume that the coefficients of the indeterminants of the first row and column are $+1$. We need to extend the idea of an auxiliary matrix given in Example \ref{bal-had-ex2}. To do this, we will follow \cite{unbiased-od} in defining the auxiliary matrix of an OD thus: Let $H$ be the Hadamard matrix obtained by setting each indeterminant of $X$ to $+1$, and label the rows of $H$ as $h_0, \dots, h_{n-1}$. Then the auxiliary matrices\Dnote{aux-note}\index{auxiliary matrices} of $X$ are given by $c_i=h_i^tx_i$. We have the following result.

\begin{lem}
Let $c_i=h_i^tx_i$, for $i \in \{0, \dots, n-1\}$, be the auxiliary matrices of an $\od(n; s_1, \dots, s_u)$ $X$ where $XX^t=\sigma I$. Then:
 \begin{defenum}
  \item\label{aux-lem-1} $\ssum_i c_ic_i^t=n \sigma I_n$, and
  \item\label{aux-lem-2} $c_ic_j^t=\delta_{ij}\sigma h_i^th_i$.
 \end{defenum}
\end{lem}
 
 We need the simple fact that if $(a,b)$ denotes the concatination of sequences $a$ and $b$, then $(a,b)$ and $(a,-b)$ is a Golay pair (see \S5). Continuing to let $c_0, \dots, c_{n-1}$ be the auxiliary matrices of the $\od(n; s_1, \dots, s_u)$ $X$, then $a=(c_0,c_1, \dots, c_{n-1},c_{n-1},\dots,c_1)$ and $b=(c_0,c_1, \dots, c_{n-1},-c_{n-1},\dots,-c_1)$ form a complementary pair. Let $A$ and $B$ be the block-circulant matrices with first rows $a$ and $b$, respectively. 
 
 Now, take $\tilde X = \left(\begin{smallmatrix} +&+\\+&- \end{smallmatrix}\right) \otimes X$ and $\tilde H = \left(\begin{smallmatrix} +&+\\+&- \end{smallmatrix}\right) \otimes H$, and label the block rows $\tilde x_0, \dots, \tilde x_{2n-1}$ and $\tilde h_0, \dots, \tilde h_{2n-1}$. Define $G=\tilde h_0^t \tilde x_0$, and define the block matrices $E^t=(\begin{smallmatrix} E_1^t & \dots & E_{2n-1}^t \end{smallmatrix})^t$ and $F=(\begin{smallmatrix} F_1 & \dots & F_{2n-1} \end{smallmatrix})$ by $E_i=h_0^t\tilde x_i$ and $F_i=\tilde h_i^t x_0$.
 
 As before, we then take $Z=\left(\begin{smallmatrix} G&F&-F\\E&A&B\\-E&B&A \end{smallmatrix}\right)$. The next result then follows.
 
 \begin{thm}\label{stable-od-thm}
  If there is an $\od(n;s_1, \dots, s_u)$, then the block matrix $Z$ is an $\od(4n^2; 4ns_1, \dots, 4ns_u)$. 
 \end{thm}
 
 \begin{proof}
  The proof amounts to checking the block entries of $ZZ^t$. 
  
  To begin, $GE_i^t = (\tilde h_0^t \tilde x_0)(h_0^t \tilde x_i)^t = \tilde h_0^t (\tilde x_0 \tilde x_i^t) h_0 = O$, hence $GE^t = EG^t = O$. Then
   $F_ic_j^t = (\tilde h_i^t x_0)(h_j^tx_j)^t = \tilde h_i^t (x_0x_j^t) h_j = \delta_{0j}\sigma \tilde h_i^t h_0$ so that
   \[
   FA^t = FB^t = \sigma \begin{pmatrix} \tilde h_1 \\ \vdots \\ \tilde h_{2n-1} \end{pmatrix}^t (\jj_{2n-1} \otimes h_0).
   \]
  We have, therefore, that $FA^t - FB^t = O$. Then the inner product between the first and second, and the first and third, block rows of $Z$ vanish. 
  
  Next, $E_iE_j^t = (h_0^t \tilde x_i)(h_0^t \tilde x_j)^t = h_0^t(\tilde x_i \tilde x_j^t)h_0 = \delta_{ij}2\sigma J_n$. It follows that $EE^t=(E_iE_j^t)=(\delta_{ij}2\sigma J_n) = 2\sigma(I_{2n-1} \otimes J_n)$.
  
  We need to examine the product $AB^t$ and in order to do that, we need to examine the cross-product correlations (see \S5). To begin, the product between the first block row and column of $A$ and $B^t$ is given by $c_0c_0^t + \ssum_{i=1}^{n-1}c_ic_i^t - \ssum_{i=1}^{n-1}c_ic_i^t = \sigma J_n$. Next, let $a$ and $b$ be two sequences of length $2n-1$ defined by 
  \begin{align*}
  a_i&=\begin{cases}
        c_i & \text{if $0 \leq i < n$, and} \\
        c_{2n-i-1} & \text{if $n \leq i < 2n-1$,}
       \end{cases} \\
  b_i&=\begin{cases}
        c_0 & \text{if $i=0$,} \\
        -c_i & \text{if $0 < i < n$, and} \\
        c_{2n-i-1} & \text{if $n \leq i < 2n-1$.}
       \end{cases}
  \end{align*}
  For $j \in \{1, \dots 2n-2\}$, we have $C_j(a,b) = \ssum_{i=0}^{2n-2}a_ib_{i+j}^t = \ssum_{i=0}^{n-1}a_ib_{i+j}^t + \ssum_{i=n}^{2n-2}a_ib_{i+j}^t$, where precisely one of the right-hand sums is nonzero by \ref{aux-lem-2}. For $j \in \{1, \dots, n-1\}$, we have that
  \begin{align*}
   \sum_{i=0}^{n-1}a_ib_{i+j}^t &= a_0b_j^t + \sum_{i=1}^{n-j-1}a_ib_{i+j}^t + \sum_{i=n-j}^{n-1}a_ib_{i+j}^t \\
   &= c_0c_j^t - \sum_{i=1}^{n-j-1} c_ic_{i+j}^t + \sum_{i=n-j}^{n-1}c_ic_{2n-j-i-1}^t \\
   &= \sum_{i=0}^{j-1}c_{n-j+i}c_{n-i-1}^t,
  \end{align*}
  and
  \begin{align*}
   \sum_{i=n}^{2n-2}a_ib_{i+j}^t &= \sum_{i=n}^{2n-j-2}a_ib_{i+j}^t + \sum_{i=2n-j-1}^{2n-2}a_ib_{i+j}^t \\
   &= \sum_{i=0}^{n-j-2}c_{n-i-1}c_{n-j-i-1}^t + c_jc_0^t - \sum_{i=1}^{j-1}c_{j-i}c_i^t \\
   &= -\sum_{i=1}^{j-1}c_{j-i}c_i^t.
  \end{align*}
  We have shown that, for $j \in \{1,\dots,n-1\}$, 
  \begin{align*}
   C_j(a,b) &= \sum_{i=0}^{j-1}c_{n-j+i}c_{n-i-1}^t - \sum_{i=1}^{j-1}c_{j-i}c_i^t \\
   &= \begin{cases}
       \sigma h_{n-i-1}^th_{n-i-1} & \text{if $j-1=2i$, for some $i$; and} \\
       -\sigma h_i^th_i & \text{if $j=2i$, for some $i$}
      \end{cases}
  \end{align*}
  
  Similarly, we find that $C_{2n-j-1}(a,b)=-C_j(a,b)$, for $j \in \{1, \dots, n-1\}$, and $C_j(b,a)=-C_j(a,b)$, for all $j \in \{1, \dots, 2n-2\}$.
  
  Putting things together, we have shown that
  \begin{align*}
   AB^t &= \ccirc(\sigma J,C_{2n-2}(a,b),\dots,C_1(a,b)) \\
   &= \ccirc(\sigma J,-C_1(a,b),\dots,-C_{n-1}(a,b),C_{n-1}(a,b),\dots,C_1(a,b)) \\
   &= \sigma\ccirc(J,-h_{n-1}^th_{n-1}, \dots, -h_{n/2}^th_{n/2},\\ 
   &\qquad\qquad\qquad\qquad\qquad h_{n/2}^th_{n/2},\dots,h_{n-1}^th_{n-1}) \text{, and} \\
   BA^t &= \sigma\ccirc(J,h_{n-1}^th_{n-1}, \dots, h_{n/2}^th_{n/2},\\ 
   &\qquad\qquad\qquad\qquad\qquad -h_{n/2}^th_{n/2},\dots,-h_{n-1}^th_{n-1}).
  \end{align*}
  It follows that the product between the second and third block rows vanishes, i.e. $-EE^t + AB^t + BA^t = O$.
  
  It remains to evaluate the block diagonal entries of $ZZ^t$. The first block row gives
  \begin{align*}
  GG^t + 2\ssum_{i=1}^{2n-1}F_iF_i^t &= 2\sigma J + 2\sigma\sum_{i=1}^{2n-1}\tilde h_i^t \tilde h_i \\
  &= 2\sigma J + 2\sigma\left(\sum_{i=0}^{2n-1}\tilde h_i^t \tilde h_i - \tilde h_0^t \tilde h_0\right) \\
  &= 2\sigma J + 2\sigma\left( 2nI - J \right) \\
  &= 4n\sigma I.
  \end{align*}
  By a similar argument about the sequences $a$ and $b$ given above, we find that
  \begin{align*}
   AA^t &= \sigma\ccirc(2nI-J,h_{n-1}^th_{n-1}, \dots, h_{n/2}^th_{n/2},\\ 
   &\qquad\qquad\qquad\qquad\qquad h_{n/2}^th_{n/2},\dots,h_{n-1}^th_{n-1}) \text{, and} \\
   BB^t &= \sigma\ccirc(2nI-J,-h_{n-1}^th_{n-1}, \dots, -h_{n/2}^th_{n/2},\\ 
   &\qquad\qquad\qquad\qquad\qquad -h_{n/2}^th_{n/2},\dots,-h_{n-1}^th_{n-1}). 
  \end{align*}
  Thus, $EE^t + AA^t + BB^t = 4\sigma I$. The proof is complete.
 \end{proof}
 
 We have shown that $Z$ is an OD. It remains to show that $Z$ admits a faithful split. This is accomplished with the next result.
 
 \begin{thm}
  Assume the OD $Z$ of the previous theorem. Then:
  \begin{defenum}
   \item\label{stable-split} The submatrices $\left(\begin{smallmatrix} F \\ A \\ B \end{smallmatrix}\right)$ and $\left(\begin{smallmatrix} -F \\ B \\ A \end{smallmatrix}\right)$ yield stable splits, and
   
   \item\label{unstable-split} The submatrices $\left(\begin{smallmatrix} E&A&B \end{smallmatrix}\right)$ and $\left(\begin{smallmatrix} -E&B&A \end{smallmatrix}\right)$ yield unstable splits.
  \end{defenum}
 \end{thm}
 
 \begin{proof}
  It suffices to show the result for one matrix of each class. Note that
  \[
  \arraycolsep=1.25pt\def\arraystretch{0.625}
   \left(\begin{array}{c}
    F \\ A \\ B
   \end{array}\right)
   \arraycolsep=1.25pt\def\arraystretch{0.625}
   \left(\begin{array}{ccc}
    F^t & A^t & B^t
   \end{array}\right)
   =
   \arraycolsep=1.25pt\def\arraystretch{0.625}
   \left(\begin{array}{ccc}
    FF^t & FA^t & FB^t \\
    AF^t & AA^t & AB^t \\
    BF^t & BA^t & BB^t
   \end{array}\right).
  \]
  It follows by the proof of the previous theorem that 
  \begin{align*}
  FF^t &= 2n\sigma I - \sigma J, \\ 
  FA^t &= FB^t = \sigma\sum_{i=1}^{2n-1}\tilde h_i^th_0, \\
  AA^t &= \sigma\ccirc(2nI-J,h_{n-1}^th_{n-1}, \dots, h_{n/2}^th_{n/2}, \\ 
   &\qquad\qquad\qquad\qquad\qquad h_{n/2}^th_{n/2},\dots,h_{n-1}^th_{n-1}) \text{, and} \\
  BB^t &= \sigma\ccirc(2nI-J,-h_{n-1}^th_{n-1}, \dots, -h_{n/2}^th_{n/2},\\ 
   &\qquad\qquad\qquad\qquad\qquad -h_{n/2}^th_{n/2},\dots,-h_{n-1}^th_{n-1}). 
  \end{align*}
  Therefore, the product between distinct rows is $\pm \sigma$, which shows \ref{stable-split}.
  
  Next,
  \[
  \arraycolsep=1.25pt\def\arraystretch{0.625}
   \left(\begin{array}{c}
    E^t \\ A^t \\ B^t
   \end{array}\right)
   \arraycolsep=1.25pt\def\arraystretch{0.625}
   \left(\begin{array}{ccc}
    E&A&B
   \end{array}\right)
   =
   \arraycolsep=1.25pt\def\arraystretch{0.625}
   \left(\begin{array}{ccc}
    E^tE & E^tA & E^tB \\
    A^tE & A^tA & A^tB \\
    B^tE & B^tA & B^tB
   \end{array}\right).
  \]
  However, $E^tE=n\sum_{i=1}^{2n-1}\tilde x_i^t \tilde x_0$ has off-diagonal entries in the set
  \[
    \{\pm x_1^{m_1} \cdots x_u^{m_u}x_1^{*m_1'} \cdots x_u^{*m_u'} : m_i,m_i' \in \N\},
  \]
  which shows \ref{unstable-split}.
 \end{proof}
 
 \begin{ex}
  Applying the construction to the $\od(2;1,1)$
  \begin{defenum}
   \item $
   \arraycolsep=1.25pt\def\arraystretch{0.625}
   \left(\begin{array}{cc}
    a & b \\ b & \bar a
   \end{array}\right),
   $
  \end{defenum}
  we obtain the balancedly splittble $\od(16;8,8)$
  \begin{defenum}[resume]
   \item $
   %\arraycolsep=1.25pt\def\arraystretch{0.625}
   \arraycolsep=2.0pt\def\arraystretch{0.5}
   \left(\begin{array}{cccccccccccccccc}
a&b&a&b&\mathbf{a}&\mathbf{b}&\mathbf{a}&\mathbf{b}&\mathbf{a}&\mathbf{b}&\bar{a}&\bar{b}&\bar{a}&\bar{b}&\bar{a}&\bar{b}\\
a&b&a&b&\mathbf{\bar{a}}&\mathbf{\bar{b}}&\mathbf{a}&\mathbf{b}&\mathbf{\bar{a}}&\mathbf{\bar{b}}&a&b&\bar{a}&\bar{b}&a&b\\
a&b&a&b&\mathbf{a}&\mathbf{b}&\mathbf{\bar{a}}&\mathbf{\bar{b}}&\mathbf{\bar{a}}&\mathbf{\bar{b}}&\bar{a}&\bar{b}&a&b&a&b\\
a&b&a&b&\mathbf{\bar{a}}&\mathbf{\bar{b}}&\mathbf{\bar{a}}&\mathbf{\bar{b}}&\mathbf{a}&\mathbf{b}&a&b&a&b&\bar{a}&\bar{b}\\
b&\bar{a}&b&\bar{a}&\mathbf{a}&\mathbf{b}&\mathbf{b}&\mathbf{\bar{a}}&\mathbf{b}&\mathbf{\bar{a}}&a&b&b&\bar{a}&\bar{b}&a\\
b&\bar{a}&b&\bar{a}&\mathbf{a}&\mathbf{b}&\mathbf{\bar{b}}&\mathbf{a}&\mathbf{\bar{b}}&\mathbf{a}&a&b&\bar{b}&a&b&\bar{a}\\
a&b&\bar{a}&\bar{b}&\mathbf{b}&\mathbf{\bar{a}}&\mathbf{a}&\mathbf{b}&\mathbf{b}&\mathbf{\bar{a}}&\bar{b}&a&a&b&b&\bar{a}\\
a&b&\bar{a}&\bar{b}&\mathbf{\bar{b}}&\mathbf{a}&\mathbf{a}&\mathbf{b}&\mathbf{\bar{b}}&\mathbf{a}&b&\bar{a}&a&b&\bar{b}&a\\
b&\bar{a}&\bar{b}&a&\mathbf{b}&\mathbf{\bar{a}}&\mathbf{b}&\mathbf{\bar{a}}&\mathbf{a}&\mathbf{b}&b&\bar{a}&\bar{b}&a&a&b\\
b&\bar{a}&\bar{b}&a&\mathbf{\bar{b}}&\mathbf{a}&\mathbf{\bar{b}}&\mathbf{a}&\mathbf{a}&\mathbf{b}&\bar{b}&a&b&\bar{a}&a&b\\
\bar{b}&a&\bar{b}&a&\mathbf{a}&\mathbf{b}&\mathbf{b}&\mathbf{\bar{a}}&\mathbf{\bar{b}}&\mathbf{a}&a&b&b&\bar{a}&b&\bar{a}\\
\bar{b}&a&\bar{b}&a&\mathbf{a}&\mathbf{b}&\mathbf{\bar{b}}&\mathbf{a}&\mathbf{b}&\mathbf{\bar{a}}&a&b&\bar{b}&a&\bar{b}&a\\
\bar{a}&\bar{b}&a&b&\mathbf{\bar{b}}&\mathbf{a}&\mathbf{a}&\mathbf{b}&\mathbf{b}&\mathbf{\bar{a}}&b&\bar{a}&a&b&b&\bar{a}\\
\bar{a}&\bar{b}&a&b&\mathbf{b}&\mathbf{\bar{a}}&\mathbf{a}&\mathbf{b}&\mathbf{\bar{b}}&\mathbf{a}&\bar{b}&a&a&b&\bar{b}&a\\
\bar{b}&a&b&\bar{a}&\mathbf{b}&\mathbf{\bar{a}}&\mathbf{\bar{b}}&\mathbf{a}&\mathbf{a}&\mathbf{b}&b&\bar{a}&b&\bar{a}&a&b\\
\bar{b}&a&b&\bar{a}&\mathbf{\bar{b}}&\mathbf{a}&\mathbf{b}&\mathbf{\bar{a}}&\mathbf{a}&\mathbf{b}&\bar{b}&a&\bar{b}&a&a&b\\
   \end{array}\right),
   $
  \end{defenum}
  where a stable vertical split is shown in bold.
 \end{ex}
 
 Finally, we see at once how Theorem \ref{stable-hadamard} is a consequence of Theorem \ref{stable-od-thm} as we can simply set the indeterminants to $+1$ to obtain the result.
 
\biblio
\end{document}
