\documentclass[../../main]{subfiles}

\begin{document}

 Balanced generalized weighing matrices arise as the extremal case of Fisher's Inequality for partial difference matrices \cite[see][]{jungnickel-kharaghani-bgw}. They are well-known to be equivalent to other important and interesting combinatorial configurations, and they have also proven to be very useful tools for iterative constructions of a number of different objects. 
 
 However, the construction of balanced generalized weighing matrices has proven to be a very difficult task. Indeed, most of the known parametric families of these objects belong to the so-called classical family. 
 
 With a stroke of luck, \cite{w-mat-construct} happened upon a new infinite family of these matrices, which in turn has engendered new and interesting questions. At the same time, an equivalence between balanced generalized weighing matrices and certain association schemes was developed. It is hoped that such an equivalence will lend itself to employing the powerful cannons of algebra to answer questions that have so far been elusive.
 
 Necessarily, then, we will start with prelimary chapters on the material that will be needed and employed throughout this essay. These are collected in Part I of the thesis. 
 
 Chapter 1 introduces balanced incomplete block designs and error-correcting codes. Block designs and related structures are studied first, through to various subdesigns and isomorphisms. Following this, both linear and nonlinear error-correcting codes are touched upon. Importantly, the Hamming and simplex codes are derived. The chapter concludes with a discussion on bounding the size of a code.
 
 Chapter 2 moves on to introduce the concept of a weighing matrix. The ideas of Chapter 1 and the first section of Chapter 2 are then synthesized with the advent of balanced weighing matrices, which are in turn generalized to arbitrary finite groups.
 
 Chapter 3 serves as a brief introduction to the theory of association schemes. Though this is a rich and interesting field, we will find that we will only need the most basic of ideas from this subject. As such, this chapter will be shorter than its antecedents.
 
 Following this survey of prelimary material, Part II will present the bulk of the results achieved during this program. 
 
 Chapter 4, the first of Part II, presents the novel idea of balancedly splittable orthogonal designs. The construction is then employed to construct various families of frames.
 
 In Chapter 5, a new family of balanced weighing matrices are constructed together with an equivalence to certain families of association schemes.
 
 Chapter 6 develops a construction of (balanced) weighing matrices that is similar to, but simpler to apply than, the known techniques. Additionally, the new equivalence between balanced weighing matrices and certain association schemes intimated above is generalized to arbitrary balanced generalized weighing matrices over a finite abelian group.
 
 The idea of being balanced with respect to a group is softened somewhat in the introduction of quasi-balancedness in Chapter 7. The idea is studied thoroughly and connected again to association schemes.
 
 The final chapter, Chapter 8, shows again the utility of balanced generalized weighing matrices in the construction of new families of optimal, constant weight codes.
 
 The terminus ad quem, it is hoped, is a fuller understanding of, and an appreciation for, the intricacies and subtleties of the subject that is combinatorial design theory.
 
\end{document}
